%
%%%%%%%%%%%%%%%%%%%%%%%%%%%%%%%%%%%%%%%%
\pagenumbering{roman}                   %serve per mettere i numeri romani
\chapter*{Introduzione}
%%%%%%%%%%%%%%%%%%%%%%%%%%%%%%%%%%%%%%%%%imposta l'intestazione di pagina
\rhead[\fancyplain{}{\bfseries
INTRODUZIONE}]{\fancyplain{}{\bfseries\thepage}}
\lhead[\fancyplain{}{\bfseries\thepage}]{\fancyplain{}{\bfseries
INTRODUZIONE}}
%%%%%%%%%%%%%%%%%%%%%%%%%%%%%%%%%%%%%%%%%aggiunge la voce Introduzione
                                        %   nell'indice
\addcontentsline{toc}{chapter}{Introduzione}

La gestione dei dati è un aspetto cruciale nello sviluppo di applicazioni web moderne, specialmente per le pubbliche amministrazioni che, tramite i dati, supportano la trasparenza e l’efficienza dei servizi. Tuttavia, molti database preesistenti, spesso realizzati con tecnologie legacy, presentano difficoltà di manutenzione, accessibilità e scalabilità che ne limitano l'efficienza. In questo contesto, la reingegnerizzazione dei database diventa fondamentale per modernizzare le infrastrutture digitali e migliorare l’esperienza utente. La presente tesi si propone di ottimizzare la struttura di un database esistente, migliorandone l’accessibilità, l’efficienza e la capacità di integrazione con nuovi servizi.

Oltre alla ristrutturazione del database, sono state sviluppate funzionalità aggiuntive come la registrazione e l'autenticazione degli utenti, la gestione dei profili e il caricamento di dati tramite file CSV e JSON. Il progetto sfrutta tecnologie moderne, inclusi PHP, Python, JavaScript e HTML, che permettono di ottenere scalabilità e sicurezza, rendendo il sistema più fruibile e sicuro per l’utente finale.

\section*{Obiettivo della tesi}
L’obiettivo principale della tesi è duplice: da una parte migliorare la struttura del database esistente per facilitarne l'uso e la gestione, dall’altra implementare un sistema di gestione utenti che includa funzionalità di registrazione, autenticazione, gestione dei profili e caricamento dati tramite file \textbf{CSV} e \textbf{JSON}. Tale progetto non solo migliora l’efficienza del sistema dati, ma garantisce anche un’esperienza utente sicura e fluida, grazie a un’interfaccia user-friendly e a protocolli di sicurezza avanzati.

\section*{Risultati raggiunti}
Il progetto ha portato a una significativa ottimizzazione della struttura dati, che ora risulta più scalabile e meno ridondante, facilitando la gestione delle informazioni. La ristrutturazione ha migliorato le prestazioni del sistema e ha permesso di integrare agevolmente le nuove funzionalità di gestione utenti e caricamento dati. Questo ha reso il sistema non solo più reattivo e sicuro, ma anche facilmente estendibile per future implementazioni.

\section*{Struttura della tesi}
Il seguito della tesi è così organizzato:
\begin{itemize}
    \item \textbf{Capitolo 1: Contesto e stato dell'arte}. Fornisce una panoramica delle problematiche legate alla gestione e alla reingegnerizzazione dei database, con un focus sulle sfide comuni e sulle soluzioni tecnologiche attualmente disponibili.
    \item \textbf{Capitolo 2: Tecnologie utilizzate}. Descrive in dettaglio le tecnologie e i linguaggi utilizzati nel progetto, inclusi PHP, JavaScript, Python e HTML, illustrandone i vantaggi e l’integrazione nel sistema.
    \item \textbf{Capitolo 3: Progettazione e implementazione}. Offre una descrizione dettagliata delle fasi di progettazione e implementazione, con particolare attenzione alla ristrutturazione del database e allo sviluppo del sistema di autenticazione, gestione utenti e caricamento dati da file CSV e JSON.
\end{itemize}
