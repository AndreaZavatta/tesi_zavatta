%%%%%%%%%%%%%%%%%%%%%%%%%%%%%%%%%%%%%%%%%12pt: grandezza carattere
                                        %a4paper: formato a4
                                        %openright: apre i capitoli a destra
                                        %twoside: serve per fare un
                                        %   documento fronteretro
                                        %report: stile tesi (oppure book)
\documentclass[12pt,a4paper,openright,twoside]{report}
%
%%%%%%%%%%%%%%%%%%%%%%%%%%%%%%%%%%%%%%%%%libreria per scrivere in italiano
\usepackage[italian]{babel}
%
%%%%%%%%%%%%%%%%%%%%%%%%%%%%%%%%%%%%%%%%%libreria per accettare i caratteri
                                        %   digitati da tastiera come è à
                                        %   si può usare anche
                                        %   \usepackage[T1]{fontenc}
                                        %   però con questa libreria
                                        %   il tempo di compilazione
                                        %   aumenta
\usepackage[utf8]{inputenc}
%
%%%%%%%%%%%%%%%%%%%%%%%%%%%%%%%%%%%%%%%%%libreria per impostare il documento
\usepackage{fancyhdr}
%
%%%%%%%%%%%%%%%%%%%%%%%%%%%%%%%%%%%%%%%%%libreria per avere l'indentazione
%%%%%%%%%%%%%%%%%%%%%%%%%%%%%%%%%%%%%%%%%   all'inizio dei capitoli, ...
\usepackage{indentfirst}
%
%%%%%%%%%libreria per mostrare le etichette
%\usepackage{showkeys}
%
%%%%%%%%%%%%%%%%%%%%%%%%%%%%%%%%%%%%%%%%%libreria per inserire grafici
\usepackage{graphicx}
%
%%%%%%%%%%%%%%%%%%%%%%%%%%%%%%%%%%%%%%%%%libreria per utilizzare font
                                        %   particolari ad esempio
                                        %   \textsc{}
\usepackage{newlfont}
%
%%%%%%%%%%%%%%%%%%%%%%%%%%%%%%%%%%%%%%%%%librerie matematiche
\usepackage{amssymb}
\usepackage{amsmath}
\usepackage{latexsym}
\usepackage{amsthm}
\usepackage{cite}
\usepackage{listings}
\usepackage{hyperref} 
\usepackage[square,numbers,sort]{natbib} 
\usepackage{xcolor}
%\bibliographystyle{unsrt}
\bibliographystyle{unsrtnat}

\lstset{
	frame=single,
	breaklines=true
}


%
\oddsidemargin=30pt \evensidemargin=20pt%impostano i margini
\hyphenation{sil-la-ba-zio-ne pa-ren-te-si}%serve per la sillabazione: tra parentesi 
					   %vanno inserite come nell'esempio le parole 
%					   %che latex non riesce a tagliare nel modo giusto andando a capo.

%
%%%%%%%%%%%%%%%%%%%%%%%%%%%%%%%%%%%%%%%%%comandi per l'impostazione
                                        %   della pagina, vedi il manuale
                                        %   della libreria fancyhdr
                                        %   per ulteriori delucidazioni
\pagestyle{fancy}\addtolength{\headwidth}{20pt}
\renewcommand{\chaptermark}[1]{\markboth{\thechapter.\ #1}{}}
\renewcommand{\sectionmark}[1]{\markright{\thesection \ #1}{}}
\rhead[\fancyplain{}{\bfseries\leftmark}]{\fancyplain{}{\bfseries\thepage}}
\cfoot{}
%%%%%%%%%%%%%%%%%%%%%%%%%%%%%%%%%%%%%%%%%
\linespread{1.3}                        %comando per impostare l'interlinea
%%%%%%%%%%%%%%%%%%%%%%%%%%%%%%%%%%%%%%%%%definisce nuovi comandi
%

%%%%%%%%%%%%%%%%%%%%%%%%%%%%%%%%%%%%%%
% Comandi Custom %
%%%%%%%%%%%%%%%%%%%%%%%%%%%%%%%%%%%%%%
\newcommand{\xstudent}{Nome dello studente}
\newcommand{\xsupervisor}{Nome del relatore}



%%%%%%%%%%%%%%%%%%%%%%%%%%%%%%%%%%%%%%
% Fine Preambolo %
% Inizio documento%
%%%%%%%%%%%%%%%%%%%%%%%%%%%%%%%%%%%%%%


\begin{document}

	%%%%%%%%%%%%%%%%%%%%%%%%%%%%%%%%%%%%%%%%
	% Scelta delle dimensioni della pagina %
	%%%%%%%%%%%%%%%%%%%%%%%%%%%%%%%%%%%%%%%%

	%\setlength{\textwidth}{13.5cm}
	%\setlength{\textheight}{19cm}
	%\setlength{\footskip}{3cm}
	
	%%%%%%%%%%%%%%%%%%%%%%%%%
	% inizio prefazione
	%
	% pagina del titolo, indice, sommario
	%%%%%%%%%%%%%%%%%%%%%%%%%
	
	\begin{titlepage}
\begin{center}
{{\Large{\textsc{Alma Mater Studiorum}}}\\
{\Large{\textsc{Universit\`a di Bologna}}} \\
{\textsc{Campus di Cesena}} \rule[0.1cm]{14cm}{0.1mm}
		\rule[0.5cm]{14cm}{0.6mm}
DIPARTIMENTO DI INFORMATICA – SCIENZA E INGEGNERIA
\color{red}Corso di Laurea Magistrale in Ingegneria e Scienze Informatiche }
\end{center}
\vspace{10mm}
\begin{center}
\color{red}{\LARGE{\bf Reingegnerizzazione di una piattaforma per la visualizzazione di Open Data del comune di Bologna
}}
\end{center}
\vspace{10mm}
\begin{center}
 {\large{ Elaborato in:\\
\color{red}TECNOLOGIE WEB\\}}   
\end{center}
\vspace{20mm}
\par
\noindent
\begin{minipage}[t]{0.47\textwidth}
{\large{\bf Relatore:\\
\color{red}Prof.\\
\xsupervisor}}
\vspace{2mm} % Spazio tra relatore e correlatori
\newline
{\large{\bf Correlatori:\\
\color{red}Dott. Giovanni Delnevo\\
Dott.ssa Chiara Ceccarini}}
\end{minipage}
\hfill
\begin{minipage}[t]{0.47\textwidth}\raggedleft
{\large{\bf Presentata da:\\
\color{red}\xstudent}}
\end{minipage}
\vspace{12mm}
\begin{center}
\color{red}{\large{\bf Sessione III\\%inserire il numero della sessione in cui ci si laurea
Anno Accademico 2024-2025}}%inserire l'anno accademico a cui si è iscritti
\end{center}
\end{titlepage}

	\clearpage{\pagestyle{empty}\cleardoublepage}
\begin{titlepage}                       %crea un ambiente libero da vincoli
                                        %   di margini e grandezza caratteri:
                                        %   si pu\`o modificare quello che si
                                        %   vuole, tanto fuori da questo
                                        %   ambiente tutto viene ristabilito
%
\thispagestyle{empty}                   %elimina il numero della pagina
\topmargin=6.5cm                        %imposta il margina superiore a 6.5cm
\raggedleft                             %incolonna la scrittura a destra
\large                                  %aumenta la grandezza del carattere
                                        %   a 14pt
\em                                     %emfatizza (corsivo) il carattere
Alla mia famiglia e ai miei amici, per esserci stati sempre.
%Per i sacrifici fatti, per avermi insegnato il valore della determinazione e per aver creduto in me anche nei momento più difficili.
%Questo traguardo è tanto vostro quanto mio.
\newpage                                %va in una pagina nuova

%
%%%%%%%%%%%%%%%%%%%%%%%%%%%%%%%%%%%%%%%%
\clearpage{\pagestyle{empty}\cleardoublepage}%non numera l'ultima pagina sinistra
\end{titlepage}
    
    %
%%%%%%%%%%%%%%%%%%%%%%%%%%%%%%%%%%%%%%%%
\pagenumbering{roman}                   %serve per mettere i numeri
%%%%%%%%%%%%%%%%%%%%%%%%%%%%%%%%%%%%%%%%%imposta l'intestazione di pagina
\rhead[\fancyplain{}{\bfseries
INTRODUZIONE}]{\fancyplain{}{\bfseries\thepage}}
%%%%%%%%%%%%%%%%%%%%%%%%%%%%%%%%%%%%%%%%%aggiunge la voce Introduzione
                                        %   nell'indice
\chapter*{Introduzione}
\addcontentsline{toc}{chapter}{Introduzione}
La crescente disponibilità di dati pubblici, spesso definiti come \textit{open data}, rappresenta oggi una risorsa fondamentale per promuovere trasparenza, partecipazione civica e innovazione. In ambito urbano, i dati pubblici sono utilizzati per migliorare l’efficienza dei servizi, ottimizzare la mobilità e pianificare politiche ambientali e sociali basate su evidenze concrete. Tuttavia, l’accesso e l’utilizzo efficace di queste informazioni richiedono strumenti tecnologici avanzati che possano rendere i dati fruibili e comprensibili anche per utenti non esperti. In tale contesto si inserisce il progetto di questa tesi, che si propone di creare una piattaforma interattiva per la gestione e visualizzazione dei dati pubblici del Comune di Bologna.

Questa piattaforma è concepita per supportare utenti con diversi livelli di competenza, dai cittadini interessati a comprendere le dinamiche della loro città, ai ricercatori e sviluppatori alla ricerca di informazioni approfondite per analisi o applicazioni specifiche. La piattaforma permette l’accesso a vari dataset tramite una dashboard intuitiva, che presenta funzionalità avanzate di visualizzazione interattiva e di analisi. Gli utenti possono esplorare le informazioni in modo dinamico tramite grafici, mappe e tabelle, rendendo più agevole l'individuazione di pattern e trend.

Il frontend è stato progettato seguendo i principi del design \textit{user-centered}, con particolare attenzione all’accessibilità e alla compatibilità cross-device, così da garantire un’esperienza utente ottimale su qualsiasi dispositivo. Il backend, realizzato con linguaggi e tecnologie consolidati, gestisce la logica applicativa e le operazioni di manipolazione dei dati, garantendo sicurezza e affidabilità anche in caso di richieste simultanee da più utenti. Particolare enfasi è stata posta sui meccanismi di autenticazione e autorizzazione, per garantire un accesso sicuro alle diverse funzionalità e proteggere le informazioni sensibili, in conformità con le normative vigenti, come il GDPR.

Inoltre, la piattaforma offre un sistema di gestione degli accessi basato sui ruoli (RBAC), che permette di differenziare i privilegi degli utenti, definendo in modo preciso chi può visualizzare, modificare o gestire i dati. Questa funzionalità è di fondamentale importanza per garantire la sicurezza delle informazioni e la sostenibilità del sistema in contesti di accesso pubblico.

La tesi è strutturata come segue: il primo capitolo fornisce una panoramica sullo stato dell'arte riguardo le piattaforme per la gestione di open data, con un approfondimento sulle sfide tecniche e di design. Il secondo capitolo esplora le tecnologie e i linguaggi utilizzati per sviluppare la piattaforma, mentre il terzo capitolo è dedicato all'analisi delle scelte progettuali per il frontend e il backend e si conclude con una discussione sulle potenziali evoluzioni della piattaforma.

In sintesi, questo progetto rappresenta un contributo alla gestione e valorizzazione degli open data per il Comune di Bologna, fornendo uno strumento capace di rendere accessibili e utili le informazioni pubbliche per una vasta gamma di utenti. La piattaforma si propone non solo come soluzione tecnica, ma come un passo verso una gestione più consapevole e partecipata delle risorse urbane.


\tableofcontents                        %crea l'indice
%%%%%%%%%%%%%%%%%%%%%%%%%%%%%%%%%%%%%%%%%imposta l'intestazione di pagina
\rhead[\fancyplain{}{\bfseries\leftmark}]{\fancyplain{}{\bfseries\thepage}}
\lhead[\fancyplain{}{\bfseries\thepage}]{\fancyplain{}{\bfseries
INDICE}}
%%%%%%%%%%%%%%%%%%%%%%%%%%%%%%%%%%%%%%%%%non numera l'ultima pagina sinistra
\clearpage{\pagestyle{empty}\cleardoublepage}
%%%%%%%%%%%%%%%%%%%%%%%%%%%%%%%%%%%%%%%%%non numera l'ultima pagina sinistra
\clearpage{\pagestyle{empty}\cleardoublepage}
%%%%%%%%%%%%%%%%%%%%%%%%%%%%%%%%%%%%%%%%%non numera l'ultima pagina sinistra
    
    

	
	%%%%%%%%%%%%%%%%%%%%%%%%%
	% inizio corpo del documento
	%
	% sequenze delle varie sezioni
	% è consigliato mantenere una struttura logica ben definita per separare le sezioni
	% si consiglia di reificare tale struttura fisicamente sul file system
	%%%%%%%%%%%%%%%%%%%%%%%%%
	

	
	% inclusione delle sezioni
	\clearpage{\pagestyle{empty}\cleardoublepage}
\chapter{Contesto e Stato dell'arte}                %crea il capitolo
%%%%%%%%%%%%%%%%%%%%%%%%%%%%%%%%%%%%%%%%%imposta l'intestazione di pagina
\lhead[\fancyplain{}{\bfseries\thepage}]{\fancyplain{}{\bfseries\rightmark}}
\pagenumbering{arabic}                  %mette i numeri arabi
La gestione efficiente dei dati è diventata un elemento cruciale per il corretto funzionamento dei moderni sistemi informativi. Con il rapido avanzamento tecnologico e la crescita continua della quantità di dati accessibili, le pubbliche amministrazioni e le organizzazioni private si trovano a fronteggiare volumi sempre più vasti e complessi di informazioni, spesso eterogenee e distribuite. In tale contesto, l’ingegnerizzazione dei sistemi di gestione dei dati rappresenta una leva fondamentale per garantire che le informazioni possano essere utilizzate in modo efficace, migliorando l’accessibilità, la scalabilità e la manutenibilità dei sistemi.

L’ingegnerizzazione dei dati è un processo strategico che comprende interventi mirati all’ottimizzazione e all’aggiornamento delle infrastrutture e delle architetture di gestione dei dati. In un’era in cui i dati costituiscono una risorsa indispensabile per supportare decisioni strategiche e promuovere la trasparenza amministrativa, questa metodologia consente di adattare i sistemi informativi alle nuove esigenze, facilitando l’integrazione di tecnologie e funzionalità innovative. Questo approccio risulta particolarmente rilevante nella gestione e nella valorizzazione degli open data.

Gli open data, ovvero i dati disponibili al pubblico, costituiscono una risorsa fondamentale per favorire la trasparenza, incoraggiare l’innovazione e stimolare la partecipazione civica.
Tuttavia, il valore degli open data dipende in larga misura dalla qualità delle infrastrutture che ne supportano la gestione e la pubblicazione: senza un’adeguata organizzazione, i dati possono risultare difficili da interpretare, frammentati o poco accessibili. Per questo motivo, molte amministrazioni pubbliche stanno ripensando le proprie infrastrutture informatiche e i processi di gestione dei dati, adottando metodologie di ingegnerizzazione orientate a migliorare la qualità e la fruibilità delle informazioni pubbliche.

Nel presente capitolo si approfondiranno il concetto di open data, le principali caratteristiche di questi dati e i benefici della loro adozione nella pubblica amministrazione. Inoltre, verranno analizzate le criticità legate alla gestione e alla fruibilità degli open data, con un focus specifico sul progetto di ingegnerizzazione di una piattaforma di visualizzazione dati. Tale piattaforma è stata sviluppata per evidenziare e rendere più comprensibili i dati resi disponibili attraverso open data, sfruttando tecniche di data visualization per facilitare l’accesso e l’interpretazione delle informazioni da parte di un pubblico più ampio.

\section{Introduzione agli Open Data}

In questa sezione viene introdotto il concetto di open data, spiegandone i principi fondamentali e i vantaggi per la pubblica amministrazione e la società civile. Vengono inoltre discusse le normative e l'evoluzione storica degli open data, sottolineando il loro ruolo nell’innovazione e nella trasparenza.

\subsection{Definizione e Principi Fondamentali}

Negli ultimi anni, la disponibilità e l’accesso ai dati hanno assunto un'importanza crescente nella società moderna, soprattutto in
ambito pubblico. Con il termine \textit{open data} si fa riferimento a un insieme di dati resi pubblicamente accessibili, che possono essere utilizzati, riutilizzati e distribuiti senza limitazioni. Gli open data permettono, infatti, un utilizzo libero da parte di chiunque, indipendentemente da restrizioni di copyright o limitazioni tecniche, con l’obiettivo di generare un valore aggiunto per la collettività \cite{Smith2018}.

Gli open data si fondano sul principio che i dati prodotti da enti pubblici e privati, soprattutto se finanziati con fondi pubblici, debbano essere resi accessibili per promuovere la trasparenza e l’innovazione. La disponibilità di tali dati consente di ampliare la conoscenza su vari aspetti della vita pubblica, come l’ambiente, la mobilità, la sanità e l’istruzione, facilitando la creazione di servizi innovativi e promuovendo il coinvolgimento attivo della cittadinanza \cite{Davies2019}. Questa apertura delle informazioni rappresenta, quindi, uno strumento fondamentale per avvicinare i cittadini alle istituzioni, favorendo un dialogo costante e stimolando la partecipazione attiva alle decisioni pubbliche.

\subsection{Vantaggi degli Open Data}

Oltre a promuovere la trasparenza, gli open data assumono un significativo valore economico. I dati aperti sono infatti una risorsa essenziale per il settore privato e la società civile, che li utilizza per sviluppare applicazioni e servizi utili alla collettività. L’obiettivo principale di queste politiche di apertura è fornire a cittadini, aziende e istituzioni l'accesso a una base informativa che favorisca lo sviluppo di soluzioni innovative, migliorando la qualità della vita e l’efficienza dei servizi pubblici. Le aziende, in particolare le start-up, possono beneficiare dell'accesso a questi dati riducendo i costi di acquisizione delle informazioni e, di conseguenza, sviluppando servizi mirati che rispondano alle esigenze locali.

Gli open data sono caratterizzati da principi chiave che ne determinano l’efficacia e l’utilità per la collettività. Uno di questi è la disponibilità, che implica che i dati siano facilmente reperibili e accessibili, preferibilmente in formati non proprietari come CSV, JSON o XML, che permettono una manipolazione e una condivisione agevole. Questo principio è strettamente legato alla riutilizzabilità: i dati devono essere distribuiti con licenze aperte, che ne consentano l’uso e la modifica senza particolari restrizioni, facilitando l’integrazione con altre fonti di dati \cite{Kitchin2014}. Inoltre, la loro interoperabilità, cioè l’adozione di standard condivisi, agevola l’integrazione di dati provenienti da contesti e fonti differenti, permettendo analisi più complete e l’integrazione in applicazioni complesse.

\subsection{Evoluzione e Normative degli Open Data}

Il movimento degli open data ha origini che risalgono agli anni ’90, quando iniziarono a diffondersi le prime politiche di trasparenza governativa. Il Freedom of Information Act (FOIA) negli Stati Uniti, e successivamente leggi simili in altri Paesi, rappresentano i primi passi verso una maggiore disponibilità dei dati pubblici. Con l’avvento delle tecnologie digitali e la crescente mole di informazioni, queste politiche si sono evolute, enfatizzando la necessità di rendere i dati non solo pubblici, ma anche fruibili digitalmente. A partire dagli anni 2000, il concetto di open data ha acquisito una dimensione globale, con iniziative promosse a livello internazionale per incoraggiare i governi a rendere disponibili dati in formati aperti e accessibili \cite{OpenDataHandbook}.

In Europa, la Direttiva 2003/98/CE sul riutilizzo delle informazioni del settore pubblico ha rappresentato un progresso significativo verso l’implementazione di politiche orientate agli open data, stimolando i Paesi membri a implementare pratiche di apertura dei dati. La \textit{Open Data Charter}, adottata da numerosi Paesi, promuove principi di accessibilità e riutilizzabilità per garantire che i dati possano essere impiegati efficacemente in analisi di big data, applicazioni digitali e iniziative civiche \cite{OpenDataCharter}.

Gli open data, pertanto, rappresentano una risorsa preziosa per la società contemporanea, ma il loro impiego richiede un impegno costante da parte delle istituzioni per garantire che siano utilizzati in modo efficace e sicuro. La gestione della qualità, dell’interoperabilità e della protezione dei dati è un fattore essenziale per massimizzare l'impatto positivo di questa risorsa sulla collettività.


\section{Open Data e Trasparenza nella Pubblica Amministrazione}
Questa sezione esplora l'importanza della trasparenza per la pubblica amministrazione, con particolare attenzione al ruolo degli open data. Vengono analizzate anche le iniziative globali e nazionali che promuovono la trasparenza e la partecipazione attiva dei cittadini.

\subsection{Il Ruolo della Trasparenza nelle Istituzioni Pubbliche}

La trasparenza è un principio fondamentale della gestione pubblica moderna e rappresenta un elemento essenziale per rafforzare la fiducia dei cittadini nelle istituzioni. Questo concetto implica che le informazioni e i dati relativi alle attività delle amministrazioni vengano resi accessibili e comprensibili, permettendo ai cittadini di monitorare le azioni e le decisioni degli enti pubblici. Gli \textit{open data} rappresentano il veicolo principale per attuare tale principio, consentendo un accesso libero e senza restrizioni a informazioni di pubblico interesse. Pubblicati in formati aperti e riutilizzabili, questi dati permettono ai cittadini di esercitare un controllo democratico sull'operato delle istituzioni e favoriscono una partecipazione attiva alla vita pubblica \cite{Davies2019}.

L’introduzione degli open data ha trasformato profondamente le relazioni tra le amministrazioni pubbliche e i cittadini, evolvendosi in parallelo ai progressi tecnologici. In Italia, la promozione della trasparenza è stata rafforzata dal Decreto Trasparenza (D.Lgs. n.33/2013), che stabilisce obblighi di pubblicazione e criteri di accessibilità per le informazioni pubbliche. Questa normativa, in linea con la Direttiva Europea sul Riutilizzo dei Dati del Settore Pubblico (2019/1024), punta a incentivare l’utilizzo e il riuso dei dati pubblici, incoraggiando iniziative civiche, accademiche e commerciali \cite{FOIAItalia}.

\subsection{Iniziative Globali e Impatti della Trasparenza}

A livello globale, il concetto di trasparenza è stato formalizzato attraverso numerose iniziative internazionali.Lanciata nel 2011, l’Open Government Partnership (OGP) ha incoraggiato i governi a livello globale ad adottare politiche di trasparenza e accesso ai dati pubblici, contribuendo a rafforzare la responsabilità amministrativa. Gli stati partecipanti all’OGP si impegnano a sviluppare standard e politiche di open data per garantire l’accesso pubblico a dati rilevanti per la comunità. In questo contesto, molti Paesi hanno istituito portali nazionali di open data, centralizzando l’accesso a informazioni chiave in settori come economia, sanità e ambiente, facilitando inoltre la comparabilità dei dati tra nazioni \cite{OpenDataCharter}.

Uno dei principali vantaggi degli open data nel contesto della trasparenza è la possibilità di promuovere una cittadinanza attiva e informata. I cittadini possono utilizzare i dati relativi alla spesa pubblica, alla gestione del territorio o alla sanità per monitorare come le risorse vengono impiegate, identificare possibili inefficienze e richiedere una gestione più responsabile. La possibilità di accedere a informazioni di interesse pubblico favorisce un controllo democratico diretto e accresce il livello di fiducia nei confronti delle istituzioni, migliorando la percezione della governance pubblica \cite{OECD2020}.

\subsection{Qualità e Usabilità dei Dati}

La qualità e l’usabilità dei dati pubblici sono requisiti essenziali affinché gli open data possano svolgere appieno il loro ruolo di strumenti di trasparenza. Affinché i dati aperti risultino realmente efficaci, devono rispettare criteri di accuratezza, completezza e aggiornamento periodico, in modo da rappresentare in maniera affidabile le attività amministrative. Dati incompleti, non aggiornati o imprecisi compromettono l’iniziativa di trasparenza e limitano la possibilità di una partecipazione consapevole da parte dei cittadini. Per questo motivo, le amministrazioni devono garantire che la raccolta e la pubblicazione dei dati rispettino standard qualitativi elevati, contribuendo a una rappresentazione fedele della realtà amministrativa \cite{Kitchin2014}.

Oltre alla qualità, l’adozione di formati aperti e standardizzati, come JSON, CSV o XML, è fondamentale per migliorare l’interoperabilità e favorire il riuso dei dati pubblici. La standardizzazione rende i dati più facilmente interpretabili e permette agli utenti di analizzarli in maniera indipendente, utilizzando strumenti di visualizzazione e analisi. Questo approccio ha incentivato lo sviluppo di ecosistemi digitali aperti, in cui cittadini, ricercatori e sviluppatori possono collaborare alla creazione di servizi e applicazioni basati su dati pubblici, contribuendo così a innovazioni a vantaggio della collettività \cite{OpenDataHandbook}.

\subsection{Coinvolgimento e Partecipazione dei Cittadini}

La trasparenza supportata dagli open data non solo consente un controllo più efficace sull'operato delle amministrazioni, ma promuove anche la partecipazione attiva dei cittadini. Attraverso piattaforme di open data, applicazioni per la gestione dei servizi comunali e strumenti di partecipazione civica, le amministrazioni possono facilitare il dialogo con la comunità e raccogliere feedback sui servizi pubblici. Questo tipo di interazione rappresenta un elemento essenziale per costruire una governance inclusiva e responsabile, in cui i cittadini possono contribuire attivamente alle politiche locali e influenzare le decisioni che riguardano la loro comunità.

Nel caso delle amministrazioni locali, i portali di open data rappresentano un ponte di comunicazione diretto con i cittadini, permettendo loro di accedere a informazioni su argomenti come la gestione del traffico, la qualità ambientale e i progetti in corso. Questo approccio facilita la trasparenza e rafforza la fiducia, creando un ambiente di dialogo in cui i cittadini possono esprimere opinioni e proposte basate su dati concreti. In questo modo, la trasparenza attraverso gli open data si traduce in un processo di partecipazione civica che promuove un maggiore coinvolgimento nella vita pubblica \cite{Janssen2012}.


\section{Problematiche e Sfide nella Gestione degli Open Data}

Gestire gli open data è un compito complesso che comporta sfide significative legate alla qualità, interoperabilità, sicurezza e sostenibilità. Le amministrazioni pubbliche, impegnate a favorire la trasparenza e semplificare l’accesso ai dati, devono superare diverse difficoltà per assicurare che i dati resi disponibili siano accurati, aggiornati e conformi agli standard di protezione della privacy. Questa sezione esplora le principali problematiche e sfide nella gestione degli open data, con una particolare attenzione alla qualità e accuratezza dei dati, agli standard di interoperabilità, alla sicurezza e privacy e alla sostenibilità delle risorse.



\subsection{Qualità dei Dati e Accuratezza}

La qualità dei dati è un elemento cruciale per garantire l'utilità degli open data. La qualità implica che i dati siano completi, accurati e aggiornati, in modo che possano rappresentare correttamente i fenomeni di interesse e fornire informazioni affidabili per gli utenti. Una delle sfide principali nella gestione degli open data è la difficoltà di assicurare che tutte le informazioni siano presenti e raccolte in maniera accurata. Dati incompleti o errati possono portare a decisioni mal informate e compromettere la credibilità delle amministrazioni che li pubblicano.

L’accuratezza si riferisce alla capacità dei dati di rappresentare fedelmente i fenomeni che descrivono. Un dataset accurato è essenziale per il processo decisionale, poiché fornisce una base solida su cui sviluppare politiche e interventi. Ad esempio, i dati sul traffico urbano devono essere aggiornati in tempo reale per permettere ai cittadini e alle amministrazioni di monitorare e pianificare gli spostamenti in modo efficiente. Tuttavia, garantire l’accuratezza e la completezza dei dati comporta numerosi ostacoli, poiché spesso le amministrazioni dispongono di risorse limitate e di strumenti non sempre adeguati per raccogliere e aggiornare i dati in modo continuo \cite{Janssen2012}.

L’aggiornamento regolare dei dati è un altro aspetto fondamentale per la loro qualità. Dati non aggiornati possono portare a decisioni basate su informazioni obsolete, che potrebbero non riflettere più la situazione attuale. La frequenza di aggiornamento varia a seconda del tipo di dato e delle sue applicazioni: per esempio, i dati ambientali devono essere aggiornati frequentemente per monitorare eventuali variazioni nelle condizioni atmosferiche o nei livelli di inquinamento. La carenza di risorse e di personale qualificato rende spesso difficile per le amministrazioni locali mantenere aggiornati i dati in modo continuo, compromettendo così la qualità complessiva delle informazioni fornite \cite{Kitchin2014}.

\subsection{Standard di Qualità e Interoperabilità}

Gli standard di qualità e interoperabilità rappresentano componenti essenziali per garantire che gli open data siano utilizzabili e integrabili con altre fonti. L’interoperabilità, in particolare, si riferisce alla capacità dei dati di essere letti, interpretati e utilizzati in modo coerente da diverse applicazioni e sistemi, facilitando così l'integrazione e l'analisi di dataset provenienti da fonti differenti.

Per promuovere l’interoperabilità, molti enti pubblici adottano formati aperti e standardizzati come JSON, XML e CSV, che sono ampiamente riconosciuti e supportati da vari strumenti di analisi. Il formato JSON, ad esempio, è particolarmente utile per rappresentare dati strutturati e complessi, consentendo una manipolazione flessibile e una facile integrazione con altre applicazioni. Tuttavia, la scelta del formato deve essere accompagnata da uno standard di metadati che faciliti l’interpretazione dei dati e ne garantisca la coerenza. I metadati descrivono informazioni essenziali come la fonte, la data di aggiornamento e le modalità di raccolta, facilitando il riuso dei dati in modo corretto e informato \cite{ComuneBologna2023}.

Il profilo DCAT-AP, adottato a livello europeo, è un esempio di standard di metadati che supporta l'interoperabilità dei dati pubblici. DCAT-AP fornisce un modello per la descrizione dei dati aperti, migliorando la coerenza e l’uniformità nella documentazione dei dataset. Altri standard internazionali, come l’\textit{ISO 19115} per i dati geografici, garantiscono una struttura di metadati dettagliata e consentono la condivisione dei dati tra diverse istituzioni. Adottare standard di qualità e interoperabilità consente alle amministrazioni di pubblicare dati che siano facilmente integrabili con altre fonti e utilizzabili in contesti applicativi diversificati \cite{Smith2018}.

Nonostante l’importanza degli standard, la loro adozione presenta sfide significative. La mancanza di protocolli comuni a livello internazionale limita la possibilità di integrare facilmente i dataset provenienti da paesi e amministrazioni diverse, ostacolando l’efficacia di analisi globali. Inoltre, implementare standard di qualità richiede investimenti in risorse e personale qualificato, che spesso rappresentano un limite per le amministrazioni locali \cite{OpenDataCharter}.

\subsection{Sicurezza, Privacy e Normative di Protezione}

La protezione della privacy e la sicurezza dei dati sono priorità fondamentali nella gestione degli open data, soprattutto quando i dataset contengono informazioni sensibili. Nonostante gli open data siano generalmente anonimi, la combinazione di diverse fonti può accrescere il rischio di re-identificazione, sollevando preoccupazioni riguardo alla protezione dei dati personali. Il GDPR (Regolamento Generale sulla Protezione dei Dati) richiede che le amministrazioni adottino misure rigorose per assicurare la riservatezza e la sicurezza delle informazioni personali. Per rispettare questi requisiti, è necessario adottare tecniche di anonimizzazione e aggregazione dei dati, come la soppressione delle variabili identificative e la pseudonimizzazione, che riducono il rischio di esposizione senza compromettere l’utilità dei dati per scopi analitici e applicativi \cite{GDPR2016}.

Tuttavia, implementare tali tecniche richiede competenze avanzate e risorse significative, che non tutte le amministrazioni possiedono. Inoltre, alcuni dati, come quelli relativi alla sanità o ai servizi sociali, richiedono un livello di protezione ancora più elevato, poiché contengono informazioni particolarmente sensibili. L'equilibrio tra la necessità di trasparenza e la protezione della privacy rimane una sfida complessa, che richiede un approccio bilanciato per evitare violazioni della privacy e garantire che i dati siano utilizzati in modo etico e sicuro \cite{Gurumurthy2019}.

Oltre alla protezione della privacy, la sicurezza dei dati è essenziale per prevenire accessi non autorizzati e abusi. Le amministrazioni devono adottare misure di sicurezza, come l’uso di protocolli di crittografia e la protezione degli accessi, per evitare che i dati possano essere manipolati o utilizzati in modo improprio. L’adozione di politiche di sicurezza rigorose è particolarmente importante per i dati di natura sensibile, come quelli legati alla sicurezza pubblica o alle infrastrutture critiche.

\subsection{Sostenibilità e Sfide Economiche}

La sostenibilità delle iniziative di open data è una questione centrale, poiché la gestione e la pubblicazione dei dati richiedono investimenti considerevoli in termini di risorse umane, finanziarie e tecnologiche. La raccolta, l’aggiornamento e la manutenzione continua dei dati comportano costi elevati, e molte amministrazioni locali, in particolare quelle di dimensioni ridotte, faticano a sostenere le spese necessarie per garantire un accesso continuo e affidabile ai dati.

Per affrontare questa problematica, molte amministrazioni stanno cercando di sviluppare modelli di collaborazione con il settore privato e con organizzazioni non governative. Il partenariato pubblico-privato (PPP) rappresenta una soluzione efficace per condividere i costi e le responsabilità nella gestione degli open data. Attraverso queste partnership, le amministrazioni possono accedere a competenze tecnologiche avanzate e a risorse finanziarie che consentono di mantenere la qualità e la continuità delle iniziative di open data \cite{McKinsey2013}.

 Tuttavia, implementare sistemi automatizzati richiede investimenti iniziali significativi e la disponibilità di infrastrutture tecnologiche avanzate. La mancanza di risorse finanziarie e di personale qualificato rappresenta quindi un ostacolo importante, che può compromettere la continuità delle iniziative di open data e limitare l'accessibilità dei dati a lungo termine.

\section{Strategie di Reingegnerizzazione del Software}

La reingegnerizzazione del software è un processo complesso e multidimensionale che mira a migliorare la struttura, le funzionalità e le prestazioni di un sistema esistente. Nella gestione degli open data, queste strategie permettono di trasformare i dati grezzi in risorse utili e accessibili per utenti e sviluppatori. Di seguito vengono analizzate alcune delle principali tecniche di reingegnerizzazione applicabili, con riferimento specifico a contesti di open data e alle loro sfide particolari.

\subsection{Automazione e Gestione Dinamica dei Dati}

Nei sistemi di open data, uno dei principali obiettivi della reingegnerizzazione è l’automazione della gestione dei dati. Grazie a questa tecnica, è possibile semplificare e accelerare il caricamento dei dati, riducendo la necessità di interventi manuali. Ad esempio, sistemi moderni per open data permettono l'importazione diretta di file in formati come CSV o JSON, che vengono automaticamente elaborati e trasformati in tabelle di database, pronte per essere interrogate e integrate nelle applicazioni.

L’automazione consente una gestione più efficiente delle operazioni, come l'aggiornamento periodico dei dati e la creazione automatica di tabelle basate sulle strutture rilevate nei file caricati. Questo approccio non solo riduce il tempo necessario per aggiornare i dati, ma anche il rischio di errori umani. In un contesto in cui i dati pubblici vengono aggiornati frequentemente, come nel caso dei dati sul traffico o sulle condizioni meteorologiche, l'automazione garantisce che i dati disponibili siano sempre aggiornati e accurati.

Oltre a migliorare la gestione operativa, l'automazione aiuta a mantenere la coerenza nei dati e a ridurre la ridondanza. Sistemi di reingegnerizzazione avanzati implementano algoritmi di deduplicazione e validazione dei dati, che rilevano eventuali incongruenze e assicurano l’integrità dei dati importati. Grazie a queste strategie, è possibile ottimizzare il database per garantire che i dati siano completi, accurati e pronti per essere utilizzati nelle applicazioni di visualizzazione e analisi.

\subsection{Interfacce Intuitive e Gestione degli Utenti}

La reingegnerizzazione del software richiede un ripensamento delle interfacce utente, in particolare per garantire che l’interazione con il sistema sia intuitiva e accessibile. La progettazione di dashboard centralizzate consente agli utenti di gestire le principali operazioni, come il caricamento dei dati, la gestione degli utenti e l’accesso a strumenti di visualizzazione, in un’unica interfaccia. Nei sistemi di open data, le interfacce devono essere intuitive anche per utenti senza competenze tecniche avanzate, poiché una delle finalità principali dei dati aperti è permetterne l’utilizzo da parte di un vasto pubblico.

Un’interfaccia ben strutturata permette anche una gestione efficace delle autorizzazioni. In molte applicazioni di open data, infatti, è fondamentale proteggere l'accesso a determinati dati sensibili e consentire l'accesso solo agli utenti autorizzati. La reingegnerizzazione delle interfacce può includere la definizione di ruoli specifici e permessi differenziati, consentendo un controllo dettagliato su chi può visualizzare, modificare o aggiungere dati nel sistema. Questo approccio aumenta la sicurezza e la conformità alle normative di gestione dei dati, garantendo al contempo un accesso sicuro alle informazioni.

Un altro aspetto cruciale è la sicurezza dei dati personali. La crittografia delle credenziali è una tecnica standard per proteggere le password e altre informazioni sensibili. Implementando algoritmi di hashing e tecniche avanzate di autenticazione, come l'autenticazione a due fattori (2FA), è possibile aumentare significativamente la sicurezza delle informazioni degli utenti e proteggere il sistema da accessi non autorizzati. Questi aspetti di sicurezza sono particolarmente rilevanti per i sistemi di open data, dove l’accessibilità deve essere bilanciata con la necessità di proteggere i dati degli utenti.

\subsection{Sicurezza dei Dati e Conformità alle Normative}

Nella reingegnerizzazione dei sistemi software, la sicurezza dei dati riveste un ruolo centrale, soprattutto in contesti che trattano open data, dove l’esposizione al pubblico richiede una protezione maggiore. Le tecniche di crittografia per proteggere i dati sia in transito che a riposo sono essenziali per garantire che le informazioni rimangano riservate. Per esempio, nei sistemi di gestione utenti, le credenziali possono essere salvaguardate con tecniche di hashing delle password, rendendo più difficile l’accesso non autorizzato anche in caso di violazione del sistema.

Oltre alla crittografia, la conformità alle normative di protezione dei dati rappresenta un requisito indispensabile. Il Regolamento Generale sulla Protezione dei Dati (GDPR) impone standard rigorosi per garantire che i dati personali siano gestiti in modo sicuro e rispettoso della privacy. La reingegnerizzazione dei sistemi per open data richiede quindi l’adozione di tecniche di anonimizzazione e pseudonimizzazione, che permettono di trattare e analizzare i dati senza esporre informazioni personali sensibili.

Inoltre, un sistema sicuro deve prevedere controlli di accesso basati su ruoli, garantendo che solo gli utenti autorizzati possano accedere a determinate informazioni. Questo approccio non solo migliora la sicurezza, ma consente anche una gestione più precisa delle autorizzazioni, necessaria per mantenere la conformità con le normative. L'adozione di protocolli di sicurezza, come TLS per la trasmissione sicura dei dati, è una pratica consolidata nei processi di reingegnerizzazione, specialmente quando il sistema gestisce dati accessibili al pubblico.

\subsection{Integrazione tramite API e Accessibilità dei Dati}

Un aspetto importante della reingegnerizzazione dei sistemi per open data è l’integrazione di API (Application Programming Interface), che permette di rendere i dati accessibili a sviluppatori e applicazioni esterne. Le API permettono di interrogare il database in tempo reale e di ottenere i dati in formati standardizzati, come JSON o XML, che possono essere facilmente utilizzati da altre applicazioni. Questo approccio rende il sistema flessibile e aumenta la sua interoperabilità con altri software, consentendo anche l’integrazione con strumenti avanzati di data visualization e analytics.

Le API RESTful sono tra le più utilizzate in contesti di open data per la loro semplicità e facilità di implementazione. Queste API permettono di accedere ai dati tramite richieste HTTP standard, consentendo una gestione sicura e flessibile delle informazioni. Una reingegnerizzazione efficace prevede la documentazione accurata delle API, facilitando l'accesso ai dati e promuovendo un uso più ampio delle informazioni pubbliche, poiché utenti e sviluppatori esterni possono facilmente integrare questi dati nei loro sistemi.

Inoltre, la creazione di API permette di aggiornare automaticamente i dati disponibili agli utenti, garantendo che siano sempre in linea con le informazioni più recenti. Questo è particolarmente utile nei sistemi che gestiscono dati dinamici, come i dati ambientali o di mobilità urbana, dove le informazioni devono essere aggiornate frequentemente per rimanere rilevanti. In questo modo, la reingegnerizzazione del sistema favorisce un accesso ai dati aggiornato e affidabile, semplificando la condivisione delle informazioni con altri servizi.

\subsection{Monitoraggio delle Prestazioni}

Nella reingegnerizzazione dei sistemi software, l’ottimizzazione delle prestazioni è un elemento cruciale, specialmente per piattaforme di open data che devono essere in grado di gestire elevati volumi di richieste simultanee e grandi quantità di dati. Garantire prestazioni ottimali richiede un insieme di strategie e tecniche mirate a migliorare la rapidità e l’efficienza del sistema, riducendo i tempi di risposta e migliorando l’esperienza complessiva degli utenti.

Una delle tecniche più comuni per migliorare le prestazioni è l'implementazione di un sistema di caching, che permette di memorizzare temporaneamente i dati più frequentemente richiesti in una memoria ad accesso rapido. Questo approccio consente di ridurre il numero di accessi diretti al database e migliorare significativamente la velocità di risposta, poiché i dati possono essere recuperati in tempi molto più brevi. Oltre al caching, l'ottimizzazione delle query del database è un'altra pratica fondamentale, che prevede l’analisi e la riscrittura delle query SQL per ridurre i tempi di elaborazione e minimizzare l’uso delle risorse di sistema.

Il bilanciamento del carico (load balancing) è un'altra tecnica avanzata utilizzata per distribuire le richieste su più server, evitando il sovraccarico su un singolo nodo e migliorando la resilienza complessiva del sistema. Con l’uso di bilanciatori di carico, i sistemi open data possono gestire in modo più efficiente picchi improvvisi di richieste, garantendo prestazioni costanti e riducendo il rischio di downtime.

Un aspetto strettamente legato all’ottimizzazione delle prestazioni è il monitoraggio continuo del sistema, che permette di rilevare eventuali colli di bottiglia e criticità prima che diventino problematiche per gli utenti. Il monitoraggio può essere effettuato utilizzando strumenti di log avanzati e dashboard di performance che forniscono una panoramica in tempo reale sulle metriche chiave del sistema, come l’utilizzo della CPU, della memoria e del tempo di risposta. Analizzando i dati di monitoraggio, i team di sviluppo possono identificare pattern di utilizzo, prevedere esigenze future e pianificare interventi mirati per mantenere elevate le prestazioni.

Inoltre, l’adozione di tecniche di monitoraggio proattivo e di automazione delle notifiche consente di implementare un sistema di allerta rapido in grado di notificare immediatamente eventuali anomalie o cali di prestazione. In questo modo, i tecnici possono intervenire tempestivamente e prevenire disservizi che potrebbero influire negativamente sull’accessibilità e l’affidabilità della piattaforma.

L’ottimizzazione delle prestazioni e il monitoraggio continuo non solo migliorano l’esperienza utente, ma sono anche essenziali per la sostenibilità e la scalabilità dei sistemi open data. Infatti, in un contesto dove la quantità e la varietà dei dati sono in continua crescita, l’adozione di pratiche di ottimizzazione e monitoraggio consente al sistema di adattarsi a un aumento delle richieste senza compromettere la qualità del servizio, garantendo al contempo una fruizione affidabile e costante delle informazioni pubbliche.


\subsection{Testing e Qualità del Software}

La reingegnerizzazione di un software implica inevitabilmente un'attenta fase di testing, necessaria per garantire la qualità e l’affidabilità del sistema. Nei sistemi di open data, il testing deve includere non solo le funzionalità di base, ma anche la sicurezza e la conformità ai requisiti di interoperabilità. I test automatici rappresentano una pratica standard nella reingegnerizzazione, poiché consentono di individuare rapidamente eventuali errori o incompatibilità e di risolverli in modo efficiente.

Oltre ai test funzionali, è importante eseguire test di carico e di stress, che permettono di valutare come il sistema risponde a un utilizzo intensivo. Per i sistemi open data, i test di carico sono particolarmente rilevanti, poiché il sistema potrebbe dover gestire un numero elevato di richieste simultanee provenienti da diversi utenti o applicazioni esterne. La reingegnerizzazione include dunque una fase approfondita di testing per assicurare che il sistema possa mantenere le prestazioni ottimali e offrire una user experience di alta qualità anche in condizioni di uso intensivo.

In sintesi, la reingegnerizzazione di un software per open data rappresenta un investimento significativo che consente di ottenere un sistema più sicuro, efficiente e scalabile. Le tecniche descritte, dall'automazione alla modularità e al testing, offrono una base solida per garantire che il sistema sia in grado di rispondere alle esigenze degli utenti e di adattarsi alle nuove tecnologie in modo continuo e sostenibile.

\section{Importanza della Visualizzazione dei Dati}

La visualizzazione dei dati svolge un ruolo fondamentale nel rendere accessibili e comprensibili informazioni complesse e voluminose. In un’epoca di crescita esponenziale nella produzione e disponibilità di dati, la capacità di rappresentare visivamente le informazioni è essenziale per facilitare il processo decisionale e la comprensione da parte di un vasto pubblico, che include sia i cittadini che i decisori pubblici.

Nel contesto degli open data, la data visualization assume un valore strategico: consente di interpretare fenomeni articolati e di promuovere la trasparenza e la partecipazione civica. Le rappresentazioni visive riducono la complessità dei dati e permettono di identificare in modo intuitivo trend, schemi e anomalie, rendendo i dataset di grandi dimensioni accessibili e significativi anche per coloro che non possiedono competenze tecniche avanzate. Questo strumento non è solo un supporto alla comunicazione, ma un elemento chiave della democrazia digitale, poiché consente ai cittadini di interagire direttamente con i dati pubblici e di partecipare informati ai processi decisionali \cite{Few2013, Davies2019}.

\subsection{Metodologie di Visualizzazione dei Dati}

Esistono diverse metodologie per la visualizzazione dei dati, ciascuna delle quali si adatta a specifici tipi di dati e obiettivi di comunicazione. Di seguito vengono esaminate le principali tecniche utilizzate per rappresentare visivamente i dati e i contesti in cui esse risultano più efficaci:

\subsubsection{Grafici e Diagrammi}
I grafici a barre, a linee, a torta e a dispersione sono tra le tecniche di visualizzazione più comuni e versatili, utilizzati per rappresentare dati quantitativi e per confrontare variabili. Ad esempio, i grafici a barre sono particolarmente utili per confrontare categorie discrete, come il tasso di partecipazione elettorale nei diversi quartieri, mentre i grafici a linee sono ideali per visualizzare tendenze temporali, come la variazione dei livelli di inquinamento nel corso dell'anno. I grafici a torta, invece, rappresentano proporzioni all’interno di un dataset, ma possono risultare meno efficaci se il numero di categorie è elevato o se le differenze tra i dati sono minime \cite{Few2013}.

Questi grafici sono ampiamente utilizzati nel contesto degli open data per rappresentare informazioni su mobilità, spesa pubblica e dati ambientali. Ad esempio, un grafico a barre può illustrare la ripartizione del bilancio comunale per settore, mentre un grafico a linee può rappresentare l’andamento dell'affluenza elettorale in un comune. La semplicità di interpretazione di questi grafici li rende particolarmente adatti per un pubblico ampio e diversificato.

\subsubsection{Mappe Geospaziali}
Le mappe geospaziali sono fondamentali per rappresentare dati che contengono componenti geografiche, come la distribuzione della popolazione, la qualità dell'aria e i livelli di traffico. Le mappe tematiche permettono di visualizzare la distribuzione spaziale delle variabili, rendendo immediatamente evidenti le differenze tra le aree geografiche. Ad esempio, una mappa di densità o \textit{heatmap} può evidenziare le aree con maggiore intensità di traffico, fornendo una rappresentazione visiva delle zone più congestionate in città \cite{Peterson2014}.

Questo tipo di rappresentazione è particolarmente utile nel contesto degli open data comunali, in quanto consente di monitorare il traffico, l’inquinamento e altri fenomeni territoriali, facilitando una comprensione immediata delle condizioni locali. Le amministrazioni comunali utilizzano spesso mappe interattive per informare i cittadini sulle condizioni della mobilità o sulla qualità dell'aria nelle diverse aree, permettendo loro di adottare misure preventive o di pianificare i propri spostamenti in modo efficiente \cite{Davies2019}.

\subsubsection{Grafici di Rete}
I grafici di rete vengono utilizzati per rappresentare le connessioni e le relazioni tra elementi diversi, come i flussi di trasporto o le interazioni tra utenti in un sistema. Questa metodologia si basa su nodi e collegamenti, ed è particolarmente utile in ambiti come l'analisi delle reti sociali, dei flussi di traffico o delle connessioni tra luoghi di interesse. Nel contesto degli open data, i grafici di rete possono essere utilizzati per visualizzare la rete di trasporto urbano o per analizzare le interazioni tra cittadini e servizi pubblici \cite{Kanza2019}.

\subsubsection{Dashboard Interattive}
Le dashboard interattive offrono una panoramica dettagliata di più variabili e consentono agli utenti di esplorare e filtrare i dati in base alle proprie esigenze. Le dashboard sono utilizzate da amministrazioni pubbliche e aziende per visualizzare dati aggiornati e per confrontare diversi indicatori. Nel contesto degli open data, le dashboard interattive permettono di accedere a informazioni dettagliate su argomenti di interesse pubblico, come la spesa pubblica, i dati sanitari o la qualità dei servizi comunali, facilitando il controllo democratico delle risorse e delle politiche pubbliche \cite{Smith2018}.

L'interattività delle dashboard consente ai cittadini di esplorare i dati in modo autonomo, filtrando le informazioni in base a parametri specifici e approfondendo i dettagli più rilevanti per le proprie esigenze. Questa personalizzazione dell’esperienza utente rende i dati più accessibili e comprensibili, incentivando una partecipazione civica più informata.

\subsubsection{Visualizzazioni Avanzate e Interattive}
Le visualizzazioni avanzate e interattive, realizzate con strumenti come \textit{D3.js}, \textit{Plotly} e \textit{Leaflet.js}, permettono di rappresentare dati complessi e multidimensionali in modo dinamico. Questi strumenti sono particolarmente efficaci per esplorare dati ad alta dimensionalità e per offrire una maggiore flessibilità nella visualizzazione, consentendo agli utenti di analizzare i dati in profondità e di personalizzare l'esplorazione delle informazioni. Ad esempio, \textit{Leaflet.js} è una libreria ampiamente utilizzata per creare mappe interattive che permettono di visualizzare dati geospaziali in modo immediato e intuitivo \cite{Wilkinson2021}.

Le visualizzazioni avanzate sono particolarmente utili per rappresentare dati pubblici complessi, come le condizioni del traffico o i dati ambientali, poiché permettono di aggiornare i dati in tempo reale e di fornire un’esperienza utente coinvolgente e personalizzata. Queste visualizzazioni facilitano una comunicazione chiara e accessibile dei dati, promuovendo la partecipazione attiva dei cittadini.

\subsection{La Visualizzazione dei Dati nel Contesto degli Open Data}

Nel contesto degli open data, la visualizzazione dei dati gioca un ruolo cruciale nel rendere accessibili le informazioni a un pubblico ampio e diversificato. Gli open data, infatti, sono rivolti non solo a esperti e sviluppatori, ma anche a cittadini che desiderano informarsi e partecipare alle decisioni pubbliche. Le rappresentazioni visive, come mappe, grafici interattivi e dashboard, permettono a tutti gli utenti di esplorare e interpretare i dati con facilità, offrendo una visione chiara e sintetica delle informazioni.

La visualizzazione facilita la trasformazione dei dati grezzi in conoscenza utile, consentendo ai cittadini di comprendere meglio fenomeni complessi come la distribuzione delle risorse pubbliche, le tendenze demografiche o le condizioni ambientali. Ad esempio, le mappe della qualità dell'aria o del traffico urbano permettono agli utenti di identificare immediatamente le zone critiche e di pianificare spostamenti o altre attività in modo informato \cite{Kitchin2014}.

\subsection{Criteri di Qualità e Accessibilità nella Visualizzazione}

Per garantire un'efficace fruizione delle informazioni, le visualizzazioni di open data devono rispettare alcuni criteri fondamentali di qualità e accessibilità. Le visualizzazioni devono essere chiare, leggibili e facilmente comprensibili. La scelta dei colori, delle forme e delle dimensioni deve quindi essere fatta con cura, per facilitare la distinzione tra le variabili e per ridurre il rischio di fraintendimenti \cite{Few2013}.

Inoltre, le visualizzazioni interattive, come le dashboard, consentono di esplorare i dati in modo personalizzato, favorendo l'accessibilità e l’inclusività. La possibilità di filtrare e navigare tra i dati permette agli utenti di accedere rapidamente alle informazioni di loro interesse, senza dover analizzare dataset complessi. In questo modo, la visualizzazione dei dati diventa uno strumento non solo di analisi, ma anche di comunicazione efficace e di democratizzazione delle informazioni \cite{OECD2020}.

La visualizzazione dei dati è uno strumento essenziale per interpretare e trasmettere informazioni in modo chiaro, rendendo i dati comprensibili e accessibili a un pubblico ampio e diversificato. Nel contesto degli open data, la visualizzazione è essenziale per promuovere la trasparenza e la partecipazione civica, consentendo ai cittadini di acquisire consapevolezza e di interagire attivamente con le informazioni pubbliche. La data visualization non solo supporta decisioni informate, ma rafforza anche il rapporto di fiducia tra le istituzioni e la comunità, incentivando un modello di governance aperto e inclusivo \cite{Few2013}.
	\clearpage{\pagestyle{empty}\cleardoublepage}

\chapter{Tecnologie e Linguaggi}

Nel corso del capitolo, verranno descritti il modello architetturale scelto, il flusso di dati tra i diversi componenti, e gli aspetti legati alla sicurezza e alla privacy, che rappresentano elementi fondamentali nella gestione di dati pubblici. Successivamente, sarà presentato l’utilizzo dei linguaggi di programmazione e delle librerie selezionate, per concludere con una panoramica sulla gestione delle autorizzazioni e controllo degli accessi.

\section{Architettura Tecnologica e Scelte di Progetto}
Questa sezione descrive l'architettura tecnologica e le principali scelte progettuali adottate nello sviluppo della dashboard per la gestione degli open data. Verrà introdotto il modello client-server scelto per garantire scalabilità e modularità, insieme ai motivi della scelta e al flusso di comunicazione tra client e server. Infine, verranno analizzati i vantaggi di questa architettura per l'efficienza e l'interattività del sistema.

\section{Linguaggi di Programmazione e Tecnologie di Sviluppo}

Per la costruzione della dashboard interattiva sono stati impiegati vari linguaggi e tecnologie, ciascuno con un ruolo specifico nello sviluppo e nella gestione delle diverse funzionalità. HTML e CSS sono utilizzati per la struttura e lo stile della pagina, mentre JavaScript e PHP sono impiegati per gestire l'interattività e la logica applicativa della piattaforma. Questa sezione descrive l'uso di ciascuno di questi strumenti, illustrandone il ruolo all'interno del progetto \cite{flanagan2011javascript} \cite{esposito2020learning}.

La scelta di questi linguaggi non solo ha reso possibile un'architettura client-server efficiente, ma ha anche consentito di costruire un'interfaccia responsiva, reattiva e facile da manutenere \cite{mdn_docs}. Ogni linguaggio è stato selezionato per le sue caratteristiche specifiche e la capacità di integrarsi con le altre componenti del sistema. Di seguito sono illustrate le funzionalità principali di ciascun linguaggio e il loro impiego nel progetto.

\subsection{HTML per la Strutturazione del Contenuto}
HTML (HyperText Markup Language) rappresenta la base per la struttura delle pagine web ed è utilizzato nella dashboard per organizzare e definire il contenuto visualizzato dall’utente.

In questo progetto, HTML fornisce una struttura semantica che facilita la navigazione e migliora l'accessibilità. Ogni elemento della dashboard è definito attraverso tag HTML che organizzano il contenuto in maniera logica \cite{robbins2012learning}. Questa organizzazione semantica consente anche di ottimizzare la compatibilità della dashboard con diversi dispositivi e screen reader, garantendo così un'esperienza utente inclusiva.

HTML è stato impiegato anche per definire i collegamenti tra le varie sezioni della dashboard e per creare componenti interattivi che gli utenti possono attivare, come bottoni e campi di input.

\subsection{CSS per lo Stile e la Responsività}
CSS (Cascading Style Sheets) è stato utilizzato per definire lo stile visivo della dashboard, creando un layout estetico e mantenendo un aspetto coerente tra le diverse sezioni della piattaforma \cite{meyer2007css}. Grazie a CSS, la dashboard può adattarsi in modo dinamico a diverse dimensioni di schermo, assicurando che l’esperienza utente sia ottimale su dispositivi desktop, tablet e smartphone.

Per raggiungere un design responsivo, sono state implementate regole di layout flessibili e media query, che consentono di modificare lo stile della dashboard in base alle caratteristiche del dispositivo utilizzato \cite{marcotte2011responsive}. Ogni sezione è stata progettata per adattarsi automaticamente alle dimensioni dello schermo, migliorando la fruibilità su dispositivi mobili. Inoltre, CSS è stato utilizzato per garantire la coerenza visiva, con uno schema di colori, font e spaziatura uniforme, che rende la dashboard intuitiva e visivamente armoniosa.

Sono state impiegate tecniche avanzate di layout CSS, come Flexbox, per creare una struttura che risponda fluidamente ai cambiamenti di dimensione e di orientamento dello schermo \cite{kevin2020flexbox}. Flexbox infatti, è stato utilizzato per disporre in maniera flessibile le sezioni principali della pagina.

% Qui potresti continuare ad aggiungere sezioni extra o esempi aggiuntivi come quelli in precedenza, per garantire la completezza e raggiungere le pagine richieste.
\subsection{JavaScript per l’Interattività}
JavaScript è stato utilizzato nel frontend per gestire l'interattività della dashboard, consentendo un aggiornamento dinamico dei contenuti in risposta alle azioni dell'utente. Grazie a JavaScript, la piattaforma offre funzionalità interattive come la gestione degli eventi, le animazioni e l'aggiornamento in tempo reale delle sezioni informative, senza richiedere il ricaricamento della pagina \cite{flanagan2011javascript}.


JavaScript è stato inoltre impiegato per la validazione dei form di input, come quelli per la registrazione e il login degli utenti. Questa validazione lato client permette di verificare che i dati inseriti siano corretti prima dell'invio al server, riducendo la necessità di gestione degli errori a livello di backend e migliorando l’esperienza utente complessiva \cite{wroblewski2011web}. Grazie a JavaScript, la dashboard supporta anche animazioni e transizioni visive, rendendo l'interazione con la piattaforma più intuitiva e coinvolgente.

\subsection{PHP per la Logica di Backend}
PHP (acronimo ricorsivo di "PHP: Hypertext Preprocessor") è stato scelto come linguaggio di backend per gestire la logica applicativa della dashboard, occupandosi dell'elaborazione delle richieste e dell'interazione con il database. Grazie a PHP, la piattaforma è in grado di elaborare le informazioni provenienti dal frontend, eseguire operazioni sui dati e rispondere alle richieste dell'utente in maniera efficiente \cite{esposito2020learning}.

In questo progetto, PHP è responsabile per la gestione di operazioni come l'autenticazione degli utenti, la creazione e modifica delle risorse, e l’elaborazione dei file caricati. Inoltre, PHP utilizza query parametrizzate per interagire con il database in modo sicuro, prevenendo vulnerabilità comuni come l'SQL injection \cite{mccool2012php}. L'adozione di PHP per la logica di backend ha permesso di costruire un sistema scalabile, in grado di gestire in modo robusto le operazioni necessarie per il corretto funzionamento della dashboard.

PHP supporta inoltre la gestione dei permessi e dei ruoli utente, consentendo di limitare l’accesso a determinate funzionalità in base ai privilegi dell’utente. Questo approccio garantisce che solo gli utenti autorizzati possano effettuare operazioni critiche, come l’aggiunta, la modifica o l’eliminazione di dati. La flessibilità di PHP nella gestione delle sessioni utente e l'integrazione con il database rendono possibile una gestione sicura e strutturata delle informazioni all'interno della piattaforma.

In sintesi, l'uso combinato di HTML, CSS, JavaScript e PHP ha permesso di creare una dashboard interattiva, reattiva e sicura, offrendo un’esperienza utente di alta qualità e un backend solido e scalabile per il trattamento dei dati.

\section{Librerie e Framework per la Visualizzazione dei Dati}

Per migliorare la capacità della dashboard di presentare dati in modo interattivo e intuitivo, sono state utilizzate diverse librerie JavaScript specializzate nella visualizzazione dei dati geografici e delle heatmap. Queste librerie permettono alla piattaforma di mostrare graficamente i dati di traffico e di votazione, offrendo una rappresentazione chiara delle aree con maggiore densità e agevolando l'analisi spaziale \cite{krzywinski2010data}.

Le principali librerie integrate nel progetto sono \texttt{heatmap.js} e \texttt{Leaflet}. Insieme, queste librerie consentono di sovrapporre dati di densità su mappe interattive, e la libreria \texttt{leaflet-heatmap.js} viene utilizzata per facilitare l’integrazione di heatmap direttamente sulle mappe di Leaflet \cite{harrower2003colorbrewer} \cite{coburn2014}. Di seguito vengono descritti in dettaglio il ruolo e il funzionamento di ciascuna di queste librerie.

\subsection{heatmap.js per Visualizzazioni di Densità}
\texttt{heatmap.js} è una libreria JavaScript progettata specificamente per creare heatmap, una tipologia di visualizzazione che permette di evidenziare le aree con maggiore intensità di dati attraverso variazioni di colore \cite{tufte1983visual}. Questa libreria si basa su tecniche di rendering tramite Canvas, rendendola particolarmente efficiente per il rendering di grandi quantità di dati. Nella dashboard, \texttt{heatmap.js} viene utilizzata per rappresentare la densità del traffico nelle diverse zone di Bologna, fornendo un’idea chiara delle aree più trafficate in base ai dati raccolti.

Il file \texttt{heatmap.js} offre una configurazione flessibile, consentendo di regolare parametri come il raggio dei punti, la gradazione dei colori e l’opacità, in modo da ottenere una visualizzazione personalizzata e chiara \cite{spence2007information}. La configurazione di default include un raggio di 40 pixel e una gradazione cromatica che passa dal blu al rosso per indicare livelli crescenti di densità. Questo approccio visivo permette di evidenziare facilmente le zone più congestionate all’interno della mappa.

L’integrazione di \texttt{heatmap.js} nel progetto consente di convertire i dati di traffico in una rappresentazione grafica facilmente interpretabile, offrendo agli utenti una panoramica immediata dei flussi di traffico. La libreria è configurata per aggiornare dinamicamente i dati in base all’intervallo di tempo selezionato dall’utente, rendendo l’esperienza utente interattiva e fluida.

\subsection{Leaflet per Mappe Interattive}
\texttt{Leaflet} è una libreria open-source leggera per la creazione di mappe interattive, supportata da numerosi plugin e compatibile con la maggior parte dei browser moderni \cite{leaflet_docs}. Nel contesto della dashboard, Leaflet svolge il ruolo fondamentale di piattaforma di visualizzazione geografica, consentendo di visualizzare mappe interattive con una grande flessibilità. Leaflet supporta diverse tipologie di tile map, ed è stato configurato per utilizzare mappe di OpenStreetMap, fornendo così una base geografica dettagliata e gratuita \cite{haklay2008openstreetmap}.

Per gestire la visualizzazione dei dati di traffico, \texttt{Leaflet} permette di aggiungere marker personalizzati che rappresentano specifici punti di interesse o zone con elevata densità di traffico. Il file \texttt{leaflet.css} definisce gli stili necessari per mantenere la consistenza grafica e il comportamento interattivo della mappa, mentre il file \texttt{leaflet.js} gestisce le funzioni per la visualizzazione e manipolazione delle mappe. Leaflet consente anche di configurare funzioni avanzate, come il controllo dello zoom, la visualizzazione di layer sovrapposti e la personalizzazione dei popup \cite{peterson2014interactive}.

In questo progetto, \texttt{Leaflet} è stato utilizzato per visualizzare la distribuzione dei dati sul traffico e i risultati delle votazioni, permettendo agli utenti di esplorare geograficamente i dataset tramite una mappa. La libreria offre inoltre funzionalità per il rendering in tempo reale, garantendo che i dati visualizzati siano aggiornati secondo le interazioni dell'utente.

\subsection{leaflet-heatmap.js per la Sovrapposizione di Heatmap su Mappe Interattive}
\texttt{leaflet-heatmap.js} è un plugin che combina le funzionalità di \texttt{heatmap.js} e \texttt{Leaflet}, permettendo di sovrapporre heatmap su mappe interattive. Questo plugin facilita la creazione di mappe tematiche che evidenziano le aree ad alta densità di traffico direttamente su una mappa di Leaflet, integrando i dati di \texttt{heatmap.js} con la struttura geografica di \texttt{Leaflet} \cite{leaflet_heatmap_docs}. Grazie a questa combinazione, gli utenti possono visualizzare le informazioni di densità in relazione alla geografia urbana di Bologna, ottenendo una rappresentazione spaziale accurata e intuitiva.

Il plugin è configurato per supportare le trasformazioni di scala e di zoom di Leaflet, garantendo che la heatmap si adatti automaticamente al livello di zoom scelto dall’utente. Questa flessibilità permette di esplorare il traffico a diverse scale geografiche, dai dettagli di quartiere fino alla visione dell’intera città. La sovrapposizione di dati viene realizzata utilizzando un \texttt{div} che contiene il layer di heatmap, posizionato sopra il layer della mappa di Leaflet, e l’aggiornamento del layer è sincronizzato con gli eventi di zoom e spostamento della mappa.

Nel contesto del progetto, \texttt{leaflet-heatmap.js} si occupa di trasformare i dati di traffico in heatmap interattive, che si aggiornano dinamicamente man mano che l’utente esplora diverse aree o cambia i filtri temporali. Il plugin si basa su un’architettura compatta e reattiva, rendendolo una soluzione efficace per la visualizzazione di dati dinamici in tempo reale \cite{bostock2012d3}.

\subsection{Vantaggi dell’Uso di Librerie per la Visualizzazione dei Dati}
L’integrazione di \texttt{heatmap.js}, \texttt{Leaflet} e \texttt{leaflet-heatmap.js} nella piattaforma permette di ottenere una rappresentazione visiva dei dati ricca e interattiva, migliorando notevolmente l'esperienza utente. Queste librerie combinano la capacità di visualizzare informazioni dettagliate sulla densità di traffico con un’interfaccia geografica intuitiva, facilitando l’analisi e la comprensione dei dati da parte degli utenti \cite{heer2012interactive}.

La visualizzazione delle heatmap sovrapposte su mappe interattive supporta un'analisi esplorativa approfondita, offrendo strumenti per rilevare pattern di densità e identificare aree chiave all'interno del territorio urbano. Questo approccio consente una comprensione più immediata delle dinamiche cittadine e facilita la comunicazione di dati complessi in modo accessibile e visivamente accattivante.

In sintesi, le librerie per la visualizzazione dei dati utilizzate nella dashboard rendono possibile una rappresentazione interattiva e dinamica delle informazioni di traffico, supportando una gestione efficace degli open data e fornendo una base per un'analisi spaziale precisa e immediata.

\section{Database e Gestione dei Dati}

Il database della piattaforma è progettato per archiviare e gestire sia i dati di traffico che i dati delle votazioni comunali, consentendo una gestione strutturata e sicura delle informazioni. La scelta di utilizzare un database relazionale risponde alla necessità di mantenere tabelle con relazioni definite, che permettono di gestire i dati in modo organizzato, ottimizzando la ricerca e la manipolazione delle informazioni \cite{connolly2014database}. Questa sezione esplora la struttura delle tabelle principali e descrive come i dati sono archiviati per supportare le funzionalità della dashboard.

\subsection{Struttura del Database per la Gestione del Traffico}
Le tabelle che gestiscono i dati di traffico consentono di memorizzare informazioni geografiche, rilevamenti di traffico e dettagli specifici relativi alle misurazioni di flusso veicolare. La struttura delle tabelle relative ai dati di traffico include:

\begin{itemize}
    \item \textbf{comuni}: Questa tabella memorizza i comuni relativi ai dati gestiti. Ogni comune è identificato da un \texttt{id} univoco e contiene attributi quali \texttt{nome} e \texttt{descrizione}.
    
    \item \textbf{vie}: Questa tabella rappresenta le vie presenti nei dati di traffico. Per ciascuna via, sono archiviati dati come \texttt{codice\_via}, \texttt{nome\_via}, \texttt{codice\_arco} (che identifica segmenti di strada specifici), \texttt{nodo\_da} e \texttt{nodo\_a} (che rappresentano i nodi di origine e destinazione), nonché \texttt{direzione}. Ogni via è associata a un \texttt{comune\_id}, che funge da chiave esterna verso la tabella dei comuni \cite{elmasri2016fundamentals}.

    \item \textbf{spire}: Questa tabella rappresenta le "spire" di rilevamento posizionate sulle strade per misurare il flusso di traffico. Ogni spira è identificata da un \texttt{id} univoco e include informazioni come \texttt{codimpsem}, \texttt{longitudine}, \texttt{latitudine}, \texttt{geopoint} e \texttt{ID\_univoco\_stazione\_spira}. Ogni spira è associata a una via tramite la chiave esterna \texttt{codice\_via}.

    \item \textbf{rilevazioni\_traffico}: Questa tabella raccoglie i dati delle rilevazioni di traffico effettuate dalle spire. Include attributi come la \texttt{data} della rilevazione, \texttt{codice\_spira} (che funge da chiave esterna verso la tabella spire), \texttt{giorno\_settimana}, \texttt{giorno}, \texttt{mese} e \texttt{anno}. Questa struttura permette di archiviare e recuperare i dati di traffico in base alla data e alla posizione \cite{silberschatz2020database}.

    \item \textbf{dettagli\_traffico}: Per ciascuna rilevazione, questa tabella memorizza i dettagli dei flussi di traffico suddivisi per ora. Gli attributi orari sono organizzati da \texttt{00:00-01:00} a \texttt{23:00-24:00} e includono anche somme per fasce temporali come \texttt{notte}, \texttt{mattina}, \texttt{pomeriggio} e \texttt{sera}. Ciascun record è collegato a una rilevazione specifica tramite la chiave esterna \texttt{id\_rilevazione}.

    \item \textbf{dettagli\_generali}: Questa tabella archivia informazioni aggiuntive per ciascuna rilevazione, come \texttt{livello} di traffico, \texttt{tipologia}, \texttt{stato} della strada, \texttt{direzione} e \texttt{angolo}. L’identificazione di ogni rilevazione avviene tramite la chiave esterna \texttt{id\_rilevazione}.
\end{itemize}

Questa struttura relazionale permette di monitorare e analizzare in dettaglio i flussi di traffico, correlando i dati raccolti con specifiche vie, comuni e spire. La suddivisione dei dati per fasce orarie e tipologie facilita l'analisi delle variazioni di traffico in base a parametri spaziali e temporali \cite{date2019introduction}.

\subsection{Struttura del Database per la Gestione delle Votazioni}
Per la gestione dei dati di votazione comunale, il database include tabelle che rappresentano i politici, le sessioni consiliari e i dettagli di presenza e votazione. La struttura delle tabelle relative alle votazioni include:

\begin{itemize}
    \item \textbf{politici}: Questa tabella memorizza le informazioni dei politici, con attributi come \texttt{nominativo} e \texttt{gruppo\_politico}. Ogni politico è identificato da un \texttt{id} univoco, che permette di collegarlo alle sessioni e ai voti.

    \item \textbf{sedute}: Rappresenta le sedute consiliari, ognuna delle quali è identificata da una \texttt{data\_seduta} univoca. La tabella memorizza il \texttt{id} di ciascuna seduta, consentendo di collegare i politici alle singole sessioni in cui hanno partecipato.

    \item \textbf{presenze}: Questa tabella registra la presenza dei politici a ogni seduta, specificando per ciascun record l'\texttt{id\_politico}, l'\texttt{id\_seduta}, e l’attributo \texttt{presenza} che indica se il politico era presente o assente. Questa struttura permette di monitorare la frequenza di partecipazione dei politici alle sedute consiliari.

    \item \textbf{votazioni}: Memorizza i dettagli delle votazioni, tra cui il numero di votazioni e la percentuale di presenza alle votazioni per ogni politico. I record di questa tabella includono un riferimento a ciascuna \texttt{presenza\_id}, collegandoli alle rispettive partecipazioni, e i campi \texttt{num\_votazioni} e \texttt{percentuale\_presenza\_alle\_votazioni} per analizzare l’attività di voto dei politici.
\end{itemize}

Questa struttura relazionale consente di monitorare la partecipazione dei politici alle sedute consiliari e alle votazioni, permettendo di generare report dettagliati e di analizzare i dati per valutare la frequenza e l'impegno di ciascun membro.

\subsection{Autenticazione e Gestione dei Permessi}
La piattaforma implementa un sistema di autenticazione e gestione delle autorizzazioni basato su un insieme di tabelle relazionate, progettate per garantire una sicurezza avanzata e controllare l’accesso a determinate funzionalità secondo il livello di autorizzazione degli utenti. Le tabelle principali per la gestione dell’autenticazione e dei permessi includono:

\begin{itemize}
    \item \textbf{users}: Questa tabella memorizza gli account degli utenti che possono accedere alla dashboard, ognuno con privilegi specifici in base al profilo assegnato. Gli attributi principali della tabella \texttt{users} includono:
        \begin{itemize}
            \item \texttt{id}: chiave primaria univoca per ciascun account utente.
            \item \texttt{username}: nome utente univoco utilizzato per il login.
            \item \texttt{password\_hash}: hash della password dell’utente, salvata in formato sicuro tramite algoritmi di hashing (ad esempio, bcrypt), per garantire la protezione delle credenziali \cite{menezes1996handbook}.
            \item \texttt{profile\_id}: chiave esterna che si riferisce alla tabella \texttt{profile}, specificando il profilo dell’utente e i permessi associati.
        \end{itemize}


    \item \textbf{profile}: Questa tabella definisce i ruoli dei diversi profili e i rispettivi livelli di autorizzazione. I principali attributi della tabella \texttt{profile} includono:
        \begin{itemize}
            \item \texttt{id}: chiave primaria per identificare ciascun profilo.
            \item \texttt{role\_name}: nome del ruolo, ad esempio "Admin", "User".
        \end{itemize}

    \item \textbf{permissions}: Questa tabella contiene l’elenco delle autorizzazioni specifiche che possono essere assegnate ai profili. Ogni record della tabella rappresenta un permesso specifico con i seguenti attributi:
        \begin{itemize}
            \item \texttt{id}: chiave primaria univoca.
            \item \texttt{permission\_name}: descrizione del permesso, ad esempio "visualizzare dati", "modificare dati", "eliminare dati".
        \end{itemize}

    \item \textbf{profile\_permissions}: Questa tabella di associazione gestisce le relazioni tra i profili e i permessi, permettendo di assegnare più permessi a un singolo profilo. Gli attributi principali includono:
        \begin{itemize}
            \item \texttt{profile\_id}: chiave esterna che si riferisce alla tabella \texttt{profile}, collegando un profilo a una serie di permessi.
            \item \texttt{permission\_id}: chiave esterna che si riferisce alla tabella \texttt{permissions}, specificando un permesso assegnato al profilo.
        \end{itemize}
\end{itemize}

\paragraph{Funzionamento del Sistema di Autenticazione e Autorizzazione}
Quando un utente tenta di accedere alla piattaforma, viene verificato tramite il proprio \texttt{username} e \texttt{password\_hash}. Se le credenziali sono corrette, l’utente viene autenticato e associato a un profilo specifico, definito tramite il campo \texttt{profile\_id}. 

Grazie alla struttura della tabella \texttt{profile\_permissions}, il sistema verifica i permessi associati al profilo dell’utente, determinando quali azioni può eseguire. Questo modello consente di definire in modo preciso i diritti di ciascun utente all'interno della dashboard, permettendo una gestione flessibile e scalabile dei ruoli e dei permessi.

La combinazione di queste tabelle permette di implementare un controllo degli accessi basato sui ruoli (Role-Based Access Control, RBAC), dove i permessi vengono associati ai profili anziché ai singoli utenti. Questa struttura favorisce una gestione centralizzata delle autorizzazioni e semplifica l’assegnazione dei permessi, specialmente in contesti in cui il numero di utenti amministrativi è elevato \cite{sandhu1996role}.


\section{Gestione delle Autorizzazioni e Controllo degli Accessi}

In questa sezione viene trattato il sistema di gestione delle autorizzazioni e del controllo degli accessi implementato nella piattaforma. Verrà descritto come l’autenticazione e l’assegnazione dei ruoli contribuiscano a garantire la sicurezza della dashboard, limitando l’accesso alle funzioni più critiche solo agli utenti autorizzati. Saranno approfonditi il modello di autenticazione e le misure adottate per la protezione dei dati sensibili.

\subsection{Modello di Autenticazione e Autorizzazione}
Il modello di autenticazione e autorizzazione adottato nella dashboard si basa su un sistema di login protetto, che verifica l’identità dell’utente attraverso l’uso di credenziali uniche (username e password) \cite{ferraiolo2003role}. Una volta autenticati, gli utenti ricevono un livello di autorizzazione che determina le funzioni disponibili nella piattaforma. Gli utenti sono suddivisi principalmente in due categorie:

\begin{itemize}
    \item \textbf{Utenti standard}: Gli utenti standard hanno accesso alle funzionalità di base della dashboard, come la visualizzazione dei dati, l’analisi dei dati di traffico e l’accesso ai grafici di visualizzazione. Non hanno, tuttavia, permessi per modificare o aggiornare le informazioni presenti nel sistema.

    \item \textbf{Amministratori}: Gli amministratori dispongono di permessi avanzati che consentono di eseguire operazioni CRUD (Create, Read, Update, Delete) sui dati e di gestire altri utenti. Essi possono caricare nuovi dati, modificare le informazioni esistenti e cancellare dati obsoleti. Inoltre, possono creare, modificare e rimuovere altri utenti, assegnando loro i ruoli appropriati in base alle necessità operative \cite{sandhu1996role}.
\end{itemize}

Il sistema di autenticazione utilizza sessioni di autenticazione sicure per mantenere l’accesso all’account fino a quando l’utente non decide di disconnettersi o fino al termine del periodo di sessione. Questa struttura garantisce una gestione sicura e controllata delle credenziali utente, riducendo il rischio di accessi non autorizzati \cite{bishop2003computer}.

\subsection{Sicurezza dei Dati in Base ai Ruoli}
Ogni operazione eseguita all'interno della dashboard è subordinata a un controllo di autorizzazione che verifica il ruolo dell’utente. Questo controllo impedisce che utenti con permessi limitati possano accedere a funzionalità riservate agli amministratori, proteggendo così i dati da modifiche non autorizzate \cite{ferraiolo2003role}. 

Per esempio, un utente standard può visualizzare i dati sul traffico e sui risultati delle votazioni, ma non può apportare alcuna modifica. Gli amministratori, invece, hanno accesso alle funzioni di modifica dei dati e alle opzioni di configurazione della piattaforma. Questa separazione dei permessi è implementata tramite un controllo lato backend, dove ogni richiesta è verificata per accertare che l'utente disponga dei permessi necessari per completare l'operazione \cite{sandhu1996role}.

\subsection{Metodi di Protezione dei Dati Sensibili}
Per salvaguardare i dati sensibili e assicurare che l'accesso alle informazioni sia riservato esclusivamente agli utenti autorizzati, sono stati implementati i seguenti metodi:

\begin{itemize}
    \item \textbf{Hashing delle password}: Tutte le password sono memorizzate nel database in forma crittografata, tramite algoritmi di hashing sicuri come bcrypt. Questo approccio assicura che, anche in caso di accesso non autorizzato al database, le password degli utenti non siano leggibili \cite{menezes1996handbook}.

    \item \textbf{Query parametrizzate}: Per prevenire attacchi di tipo SQL injection, tutte le interazioni con il database utilizzano query parametrizzate, separando i dati dalle istruzioni SQL e impedendo che input malevoli possano alterare la struttura delle query. Questa tecnica è applicata sia per le operazioni di login, sia per le modifiche e gli aggiornamenti dei dati \cite{halfond2006classification}.
\end{itemize}

Queste misure di sicurezza sono integrate nel sistema per proteggere i dati e garantire che le operazioni siano eseguite solo da utenti autorizzati.


\subsection{Vantaggi della Gestione delle Autorizzazioni e del Controllo degli Accessi}
La gestione dei ruoli e il controllo degli accessi rappresentano componenti fondamentali per la sicurezza e l'integrità di un sistema che tratta dati sensibili. Implementare un sistema di controllo basato sui ruoli, o \textit{Role-Based Access Control} (RBAC), offre diversi vantaggi per la piattaforma, migliorando la protezione dei dati, la modularità della gestione e l'efficacia operativa.

Uno dei principali benefici della gestione dei ruoli è la capacità di separare i permessi in base alle responsabilità degli utenti, garantendo che solo coloro con le qualifiche necessarie possano eseguire determinate azioni. Ad esempio, un utente con un profilo di tipo "User" potrà visualizzare i dati ma non modificarli o eliminarli, riducendo il rischio di modifiche accidentali o malintenzionate. Questo approccio minimizza anche la probabilità che le informazioni sensibili siano esposte o alterate da utenti non autorizzati. La divisione dei permessi in base ai ruoli consente, inoltre, di mantenere una chiara suddivisione delle responsabilità, migliorando il controllo e la tracciabilità delle operazioni \cite{sandhu1996role}.

\paragraph{Modularità e Scalabilità}
Un sistema di controllo degli accessi basato sui ruoli rende la piattaforma altamente modulare e scalabile. Con RBAC, i permessi possono essere assegnati a gruppi di utenti piuttosto che a singoli individui, permettendo di definire con precisione i diritti associati a ciascun ruolo (ad esempio, "Admin", "User") e di gestire in modo centralizzato l'accesso alle risorse del sistema. Questa struttura è particolarmente vantaggiosa in contesti in cui gli utenti possono cambiare mansioni o entrare e uscire frequentemente dal sistema: anziché ridefinire i permessi per ogni singolo utente, è sufficiente assegnare o rimuovere il ruolo appropriato. Questo approccio riduce notevolmente il carico amministrativo, poiché le modifiche possono essere applicate a livello di gruppo, rendendo il sistema più efficiente e meno incline a errori di configurazione \cite{ferraiolo2001role}.

\paragraph{Protezione contro Account Compromessi}
Un altro importante vantaggio di RBAC è la limitazione dei danni potenziali in caso di compromissione di un account utente. Se un malintenzionato riesce a ottenere le credenziali di accesso di un utente con permessi limitati, l’accesso che ottiene è anch’esso limitato alle sole operazioni previste per quel ruolo. Ad esempio, un utente compromesso con permessi di sola visualizzazione non sarà in grado di modificare, eliminare o aggiungere dati sensibili, riducendo così il rischio di danni gravi alla piattaforma. In questo modo, RBAC rappresenta una forma di sicurezza "a strati", dove anche in caso di violazione, i permessi limitati mitigano i potenziali effetti dannosi \cite{bonneau2012quest}.

\paragraph{Controllo Dettagliato delle Autorizzazioni}
RBAC permette inoltre un controllo molto dettagliato delle autorizzazioni, garantendo una gestione precisa e rigorosa delle funzionalità accessibili per ciascun ruolo. Ogni operazione importante, come l’accesso a determinate sezioni del database o l’autorizzazione per modifiche specifiche, può essere regolata in base al livello di autorizzazione del profilo utente. Per la dashboard degli open data, questa struttura è particolarmente utile in quanto consente di distinguere chiaramente chi ha diritto di visualizzare i dati, chi può modificarli e chi ha il permesso di gestire interamente la piattaforma. \cite{sandhu1996role}

\paragraph{Audit e Tracciabilità delle Operazioni}
Grazie alla gestione dei ruoli e dei permessi, è possibile tracciare in modo dettagliato le attività eseguite da ciascun utente, registrando le azioni chiave e associandole al profilo assegnato. In questo modo, eventuali azioni non autorizzate o errori possono essere facilmente identificati e attribuiti, semplificando i processi di audit. La registrazione delle attività migliora la responsabilità degli utenti e facilita il monitoraggio continuo del sistema, permettendo di individuare comportamenti sospetti in tempo reale. \cite{davidson2016role}

\paragraph{Riduzione degli Errori e Maggiore Fiducia degli Utenti}
La gestione accurata delle autorizzazioni riduce anche la probabilità di errori operativi. Con i permessi ben delimitati per ciascun ruolo, gli utenti hanno accesso solo alle funzionalità di cui necessitano per svolgere le proprie mansioni. Ciò elimina ambiguità e riduce le possibilità di eseguire per errore operazioni non consentite. Questo aspetto è cruciale per piattaforme come una dashboard di open data, in cui la precisione e la protezione dei dati pubblici sono prioritarie. La separazione dei permessi, dunque, contribuisce a migliorare la fiducia degli utenti nell’utilizzo della piattaforma, garantendo che le operazioni siano svolte con sicurezza e nel rispetto delle normative di accesso.\cite{wright2017rbac}

\paragraph{Rispetto delle Normative di Sicurezza}
Il modello di gestione degli accessi basato sui ruoli è inoltre conforme alle best practice e normative di sicurezza che richiedono una separazione chiara dei privilegi, come il GDPR per la protezione dei dati \cite{gdpr2016regulation}. Adottare una struttura RBAC non solo aiuta a proteggere i dati sensibili, ma semplifica anche la conformità con le normative di settore, riducendo il rischio di sanzioni per non conformità.

In sintesi, l’approccio alla gestione delle autorizzazioni e al controllo degli accessi fornisce una solida base di sicurezza per la piattaforma. Grazie a un sistema di permessi modulare e scalabile, il controllo delle attività degli utenti e la protezione contro account compromessi, RBAC rappresenta una strategia di sicurezza robusta che migliora l’efficienza e la protezione dei dati, aumentando al contempo la fiducia degli utenti nel sistema.


\section{Sicurezza e Integrità dei Dati} 
Questa sezione esplora le strategie di sicurezza e integrità dei dati adottate nella piattaforma. Verranno illustrati i meccanismi per proteggere i dati da accessi non autorizzati, garantire la loro accuratezza e prevenire possibili compromissioni. Inoltre, saranno proposte prospettive di miglioramento per ottimizzare ulteriormente la protezione della dashboard.

\subsection{Prospettive di Miglioramento della Sicurezza}

Attualmente, il sistema non esegue verifiche periodiche sugli accessi e sulle operazioni degli utenti; tuttavia, la piattaforma è progettata con una struttura che consente di aggiungere funzionalità di monitoraggio e controllo in futuro. Per migliorare la sicurezza, si potrebbero implementare controlli automatici sui log di accesso per identificare attività sospette o non autorizzate, inviando notifiche agli amministratori in caso di tentativi di accesso non riusciti o di comportamenti anomali.

Un'altra potenziale estensione della piattaforma è l'implementazione della rotazione regolare delle credenziali di accesso per gli amministratori, incentivando la modifica delle password con frequenza. Questa pratica aiuterebbe a mantenere il sistema protetto da eventuali attacchi e garantirebbe che l’accesso ai dati sia costantemente controllato \cite{bishop2003computer}.

Inoltre, sarebbe possibile introdurre un sistema di autenticazione a più fattori (MFA) per gli utenti con privilegi amministrativi, aumentando ulteriormente la sicurezza degli accessi. Questa tecnica richiede all'utente di fornire almeno due forme di verifica prima di accedere, ad esempio una password e un codice generato su un dispositivo mobile \cite{bonneau2012quest}.

Un’altra possibile estensione è rappresentata dall’uso di tecniche di cifratura avanzate come la crittografia omomorfica, che permetterebbe di eseguire calcoli sui dati cifrati senza mai doverli decrittare, aumentando la sicurezza della privacy \cite{gentry2009fully}.




	\clearpage{\pagestyle{empty}\cleardoublepage}
\chapter{Analisi e Sviluppo del Frontend e Backend della Piattaforma}

Questo capitolo descrive i principi progettuali e le scelte tecniche alla base dello sviluppo del frontend e del backend della piattaforma. L'obiettivo primario è stato quello di costruire un sistema stabile, accessibile e scalabile, capace di soddisfare le necessità di una vasta gamma di utenti che accedono e interagiscono con i dati pubblici forniti dal Comune di Bologna.

La sezione di frontend si concentra sull’interfaccia utente e sulla user experience (UX), evidenziando le strategie adottate per garantire un'interazione intuitiva e flessibile. In particolare, viene approfondita la struttura della dashboard, la gestione della navigazione e le tecniche di ottimizzazione dell’interfaccia per dispositivi mobili, al fine di garantire un accesso efficace e una visualizzazione chiara dei dati.

La parte backend esplora invece i meccanismi di gestione dei dati e la logica operativa del sistema. In essa vengono descritte le tecniche di connessione e sincronizzazione con il database, la gestione degli accessi e delle autorizzazioni, nonché le soluzioni adottate per garantire la sicurezza e l'affidabilità del sistema. Il backend costituisce il fulcro delle operazioni della piattaforma, essendo responsabile dell'elaborazione dei dati e della loro disponibilità in tempo reale.

Infine, vengono discussi alcuni possibili miglioramenti e ottimizzazioni sia per il frontend sia per il backend. Questi includono l'adozione di framework moderni, l’integrazione di strumenti avanzati di logging e testing, e l'implementazione di sistemi per la gestione automatica delle connessioni e delle notifiche real-time. Questi miglioramenti potrebbero contribuire a rendere la piattaforma ancora più performante, manutenibile e sicura, consentendo una crescita sostenibile e adattandosi a nuove esigenze operative e tecnologiche.

Il capitolo è organizzato come segue:
\begin{itemize}
    \item La prima sezione analizza i requisiti e gli obiettivi specifici per il progetto, illustrando le necessità tecniche e funzionali che guidano lo sviluppo della piattaforma.
    \item Successivamente, viene presentata la struttura del frontend, con un focus su design, organizzazione dell’interfaccia e strategie per la responsività.
    \item La sezione dedicata al backend descrive la logica di gestione dei dati, i sistemi di sincronizzazione e i metodi di autenticazione, concentrandosi sulla stabilità e sicurezza del sistema.
    \item Infine, vengono discussi i miglioramenti architetturali possibili per entrambi i componenti, delineando le future direzioni di sviluppo per aumentare l’efficienza e la scalabilità della piattaforma.
\end{itemize}

Questa struttura consente di comprendere a fondo le scelte tecniche e architetturali che supportano il funzionamento della piattaforma, offrendo una panoramica completa dello sviluppo sia del frontend che del backend in un contesto di gestione e visualizzazione dei dati open data.

\section{Analisi dei Requisiti e Definizione degli Obiettivi del Progetto}
Questa sezione esamina i requisiti essenziali e gli obiettivi specifici per lo sviluppo della piattaforma. Una comprensione accurata delle necessità degli utenti e delle finalità del progetto ha guidato la definizione delle funzionalità principali e delle specifiche tecniche della piattaforma. Gli obiettivi della dashboard, i requisiti dell’interfaccia utente e i vincoli tecnici identificati formano la base per progettare un sistema capace di soddisfare le esigenze di un ampio spettro di utenti, garantendo sicurezza, affidabilità e accessibilità.

\subsection{Requisiti di Sistema e Obiettivi della Dashboard}

L'analisi dei requisiti rappresenta un passaggio essenziale nello sviluppo della piattaforma, poiché definisce le specifiche tecniche e funzionali necessarie per rispondere in modo efficace alle esigenze di un'ampia gamma di utenti, garantendo al contempo il raggiungimento degli obiettivi del Comune di Bologna. La piattaforma si pone come obiettivo la realizzazione di una dashboard inclusiva e intuitiva, capace di adattarsi alle competenze diversificate dei suoi utilizzatori, supportando un sistema di gestione dati avanzato che favorisca l’accesso a informazioni chiave per cittadini, ricercatori, sviluppatori e rappresentanti del settore pubblico \cite{nielsen1994, cooper2014}.

La diversità di utenti implica che la piattaforma debba essere progettata per supportare differenti livelli di competenza tecnica, offrendo funzionalità accessibili per i cittadini, che necessitano di un'interfaccia semplice e intuitiva, e opzioni più avanzate per sviluppatori e ricercatori, che possono richiedere strumenti complessi per l’analisi dei dati. Per soddisfare tali requisiti, la dashboard è stata sviluppata con l’obiettivo di offrire una rappresentazione chiara e immediata delle informazioni. La struttura della piattaforma è studiata per facilitare l’accesso rapido ai dati e agli strumenti di visualizzazione, garantendo al contempo una flessibilità tale da poter incorporare nuovi dataset e funzioni senza compromettere l’usabilità del sistema \cite{shneiderman2016designing}.

I requisiti funzionali stabiliti sono stati suddivisi in una serie di funzioni chiave, che rappresentano i pilastri della piattaforma:
\begin{itemize}
    \item \textbf{Visualizzazione interattiva dei dati}: la piattaforma offre diversi strumenti di visualizzazione, tra cui grafici, mappe interattive e tabelle, che permettono agli utenti di esplorare e analizzare i dati in modo intuitivo e coinvolgente. La visualizzazione dei dati è pensata per essere flessibile e per adattarsi alle diverse tipologie di dataset gestiti, agevolando l’individuazione di pattern e tendenze.
    
    \item \textbf{Gestione degli utenti e controllo degli accessi}: la piattaforma consente agli amministratori di registrare nuovi utenti, modificare le informazioni di profili esistenti e, se necessario, rimuovere utenti dal sistema. Questo sistema di gestione utenti è integrato con un controllo degli accessi basato sui permessi, che garantisce che solo determinati profili abbiano accesso a specifiche funzioni. In particolare, i dati sensibili e le funzionalità amministrative sono accessibili esclusivamente agli utenti con permessi elevati, migliorando così la sicurezza complessiva della piattaforma.
    
    \item \textbf{Importazione e aggiornamento dei dati}: per mantenere la piattaforma aggiornata e rilevante, è fondamentale che i dataset possano essere aggiornati periodicamente. Gli amministratori hanno accesso a una funzionalità di \textit{loading} che permette loro di caricare nuovi dati o aggiornare i dataset esistenti in modo semplice e veloce. Questa funzionalità è pensata per minimizzare i tempi di inattività e assicurare che gli utenti accedano sempre a dati aggiornati e accurati.
    
\end{itemize}

Queste funzionalità sono state progettate per garantire un'interazione fluida e intuitiva, pur mantenendo la possibilità di espandere il sistema in futuro. Poiché la piattaforma non si aggiorna automaticamente in tempo reale, ma dipende dall’azione degli amministratori per il caricamento dei dati, il sistema è strutturato in modo da ridurre al minimo i tempi di aggiornamento e massimizzare l'efficienza nella gestione dei dati. Questo approccio contribuisce a ridurre la necessità di interventi frequenti da parte degli amministratori, migliorando l'efficienza complessiva.

Dal punto di vista dei requisiti non funzionali, la piattaforma è stata progettata per garantire elevate prestazioni anche in condizioni di carico elevato, assicurando che il sistema rispetti standard elevati di affidabilità e possa sostenere future espansioni senza comprometterne la stabilità \cite{feldmann2002scalability}. A questo scopo, l'architettura della piattaforma è stata suddivisa in moduli indipendenti, con una chiara separazione tra le funzionalità backend e frontend. Questa organizzazione permette di mantenere la coerenza del sistema anche in caso di modifiche parziali, facilitando inoltre l’integrazione di nuove tecnologie e funzionalità.

L'architettura della piattaforma permette di aggiungere nuove funzionalità e servizi con modifiche minime al codice di base, garantendo la sostenibilità della piattaforma nel lungo periodo. La flessibilità della struttura rende infatti possibile l’espansione delle funzionalità esistenti, come l'integrazione di nuove visualizzazioni di dati, il supporto per ulteriori formati di file e l’aggiunta di funzioni di analisi avanzate. Inoltre, il sistema è progettato per garantire la massima compatibilità con standard e formati open data, permettendo al Comune di Bologna di aggiornare i dataset secondo le normative vigenti senza necessità di ulteriori personalizzazioni.

In sintesi, i requisiti di sistema e gli obiettivi della dashboard rappresentano una solida base per la creazione di una piattaforma flessibile, scalabile e inclusiva, in grado di rispondere alle esigenze di un’utenza eterogenea e di adattarsi a futuri sviluppi tecnologici e operativi. Questa pianificazione ha permesso di costruire una struttura affidabile che facilita non solo l’accesso ai dati, ma anche la loro analisi e gestione, contribuendo a valorizzare il patrimonio informativo del Comune di Bologna.

\subsection{Requisiti dell’Interfaccia Utente e User Experience (UX)}

La progettazione dell'interfaccia utente (UI) della piattaforma segue un approccio incentrato sull'utente, con particolare attenzione all'accessibilità e all'usabilità \cite{w3c2018}. L'obiettivo è quello di realizzare un'interfaccia che risponda efficacemente alle esigenze di un pubblico diversificato, che include cittadini, sviluppatori, ricercatori e funzionari pubblici, e che al contempo sia esteticamente piacevole, intuitiva e facilmente navigabile. 

I requisiti funzionali dell'interfaccia utente sono stati definiti per offrire un'esperienza ottimale, facilitando l'accesso rapido e preciso alle informazioni attraverso funzionalità chiave:
\begin{itemize}
    \item \textbf{Navigazione semplificata}: la presenza di una barra di navigazione persistente assicura un accesso rapido e continuo a tutte le sezioni principali della piattaforma. La barra di navigazione offre una visione chiara delle funzionalità disponibili e guida l’utente verso le aree di interesse con un minimo di clic, migliorando l'efficienza della navigazione e l’esperienza complessiva dell’utente \cite{cooper2014}. 
    \item \textbf{Interazione intuitiva}: l’organizzazione visiva degli elementi, è stata studiata per consentire un accesso rapido anche agli utenti meno esperti. Gli elementi interattivi sono posizionati in modo strategico, con etichette e descrizioni chiare che migliorano l'orientamento all'interno della piattaforma. Questo layout riduce il tempo necessario per apprendere e utilizzare il sistema e contribuisce a rendere l’interazione più fluida e senza errori.
\end{itemize}

La scelta di una palette cromatica uniforme e di elementi grafici chiari mira alla riduzione del carico cognitivo, facilitando la comprensione e la navigazione tra le sezioni \cite{nielsen1994}. I colori sono stati scelti per garantire un contrasto adeguato, favorendo la leggibilità e rispettando le linee guida sull’accessibilità. Font leggibili e di dimensione appropriata completano il design, rendendo l’esperienza visiva gradevole e senza sforzo. 

Inoltre, la piattaforma fornisce \textbf{feedback visivi} per ogni azione, come il caricamento dei dati o la modifica di un utente, per informare l'utente dello stato delle operazioni. Questi segnali, sotto forma di messaggi di stato o indicatori di progresso, riducono l'incertezza e contribuiscono a una maggiore confidenza nell'utilizzo della piattaforma. In particolare, i messaggi di conferma o di errore sono studiati per essere facilmente comprensibili, offrendo suggerimenti su come procedere in caso di errore.

Un’altra caratteristica chiave è la compatibilità dell'interfaccia con dispositivi mobili. L'interfaccia è stata progettata con un \textbf{approccio responsive}, adattandosi in modo dinamico a diversi dispositivi e risoluzioni di schermo, tra cui tablet e smartphone. Questo approccio garantisce che la piattaforma possa essere utilizzata da qualsiasi dispositivo, mantenendo una UX coerente e intuitiva indipendentemente dal formato dello schermo. 


\subsection{Vincoli Tecnici e Limiti Operativi}

Durante la progettazione della piattaforma, sono emersi diversi vincoli tecnici e operativi che hanno influenzato le scelte architetturali e tecnologiche. Questi vincoli hanno richiesto un'accurata pianificazione per garantire la stabilità, la sicurezza e l'efficienza del sistema. Alcuni dei vincoli principali includono:

\textbf{Compatibilità cross-browser e multi-dispositivo}. Poiché gli utenti accedono alla piattaforma da una varietà di dispositivi, inclusi desktop, tablet e smartphone, e utilizzano browser diversi come Chrome, Firefox, Safari ed Edge, è fondamentale che il sistema sia compatibile con tutte queste configurazioni. Per assicurare una visualizzazione uniforme e una navigazione fluida, è stato adottato un design responsive e testato su diverse risoluzioni di schermo e versioni di browser \cite{w3c2018}.


\textbf{Sicurezza e protezione dei dati}. La sicurezza dei dati è un aspetto cruciale per la piattaforma, soprattutto considerando che si tratta di un sistema open data accessibile al pubblico. È stato implementato un sistema di autenticazione per limitare l’accesso ad alcune funzionalità amministrative, garantendo che solo utenti autorizzati possano accedere a funzioni sensibili come la gestione dei dataset e la modifica dei profili utente.\cite{openssl2020}.

\textbf{Scalabilità della piattaforma}. Un altro aspetto operativo fondamentale è la scalabilità del sistema. La piattaforma è stata progettata per essere espandibile, permettendo di aggiungere nuove funzionalità e dataset senza stravolgere l'architettura di base. Ad esempio, in futuro sarà possibile integrare nuove viste o moduli per la visualizzazione di altri tipi di dati pubblici, semplicemente aggiungendo componenti o sezioni al sistema esistente \cite{feldmann2002scalability}. Questo approccio modulare riduce il rischio di dover eseguire riscritture estese del codice, facilitando la crescita della piattaforma in modo controllato.


In sintesi, l’analisi dei requisiti e dei vincoli operativi ha guidato la progettazione di una piattaforma robusta e scalabile, capace di rispondere efficacemente alle esigenze degli utenti e di adattarsi a futuri sviluppi. La considerazione di tali vincoli ha contribuito a costruire una base solida, che garantisce l’affidabilità e la sostenibilità della piattaforma a lungo termine.

\section{Struttura e Organizzazione del Frontend}

Il frontend della piattaforma è l'elemento visibile e interattivo con cui gli utenti finali interagiscono. Questa sezione descrive in dettaglio l'architettura dell'interfaccia utente e le strategie di organizzazione e navigazione che permettono un'interazione fluida e intuitiva. Si analizzano inoltre i principi di responsività e adattabilità che permettono alla piattaforma di essere utilizzata su diversi dispositivi, garantendo un'esperienza di utilizzo coerente e performante.

\subsection{Layout della Dashboard e Organizzazione delle Sezioni}
Il frontend della piattaforma è stato progettato con un focus specifico sull’organizzazione e la navigabilità, adottando una struttura che facilita l’accesso alle informazioni e supporta la visualizzazione interattiva dei dati in modo dinamico e intuitivo \cite{cooper2014}. La scelta di un layout modulare e di tecnologie che garantiscano una piena responsività ha permesso di rendere la piattaforma fruibile sia da desktop che da dispositivi mobili \cite{w3c2018}.

La dashboard è stata progettata seguendo i principi di design modulare, una scelta che consente di organizzare i componenti in sezioni indipendenti, ciascuna dedicata a una funzionalità o esigenza specifica degli utenti \cite{shneiderman2016designing}. Questa organizzazione ha permesso di ottenere un’interfaccia chiara e strutturata, in cui ogni componente è facilmente accessibile e visualizzabile \cite{nielsen1994}.

\textbf{Sezioni principali e organizzazione degli spazi}. Il layout della dashboard è organizzato in due aree principali: una barra di navigazione laterale e un'area centrale che cambia contenuto in base alla selezione effettuata nella barra di navigazione. La barra di navigazione, sempre visibile sul lato sinistro, offre accesso rapido a tutte le funzionalità principali, tra cui il caricamento dei dataset, la visualizzazione grafica dei dati delle applicazioni, la registrazione di nuovi utenti e la gestione degli utenti (modifica e cancellazione). L’accesso a queste funzionalità è gestito in base ai privilegi dell’utente, con alcune opzioni riservate esclusivamente agli amministratori. Questo design facilita la navigazione, consentendo anche agli utenti meno esperti di trovare rapidamente le opzioni necessarie, riducendo i tempi di ricerca e migliorando l’efficienza complessiva \cite{feldmann2002scalability}.


\textbf{Area di visualizzazione dei dati}. L’area centrale della dashboard è configurata per ospitare diverse funzionalità, tra cui una sezione dedicata alla visualizzazione dei dati selezionata tramite il menu di navigazione. Nella sezione di \textit{Data Visualization}, l’utente può scegliere un dataset specifico e visualizzarne i dati in modo intuitivo e interattivo. Per esempio, i dati relativi al traffico urbano possono essere rappresentati su una mappa interattiva, consentendo di analizzare facilmente la distribuzione geografica degli eventi. Allo stesso modo, i dati riguardanti le presenze e le votazioni del consiglio comunale di Bologna possono essere visualizzati attraverso grafici, diagrammi e tabelle che permettono di identificare rapidamente tendenze e confronti. Questa disposizione centrale delle funzionalità facilita la fruizione e l’esplorazione dei dati selezionati, rendendo le informazioni accessibili e intuitive.

\textbf{Menu di selezione dataset}. All’interno della sezione di \textit{Data Visualization}, l’utente può selezionare rapidamente il dataset che intende visualizzare tramite un menu a tendina. Dopo aver scelto il dataset desiderato, la visualizzazione viene aggiornata automaticamente, mostrando le informazioni nel formato più appropriato, come mappe per i dati geografici o grafici e tabelle per dati statistici. Questo approccio flessibile consente di esplorare i dati in base alle necessità dell'utente e di adattare la visualizzazione alle caratteristiche specifiche del dataset, semplificando il processo di analisi e rendendo i dati più facilmente comprensibili.


\textbf{Design modulare e flessibilità}. La modularità del layout consente inoltre di adattare e aggiornare facilmente la dashboard, aggiungendo nuove sezioni o funzionalità senza alterare la struttura generale. Questo approccio rende la piattaforma più flessibile, consentendo di rispondere alle esigenze future degli utenti senza dover riorganizzare l'interfaccia in modo complesso. La struttura modulare del layout supporta così una gestione scalabile e una manutenzione semplificata.

\subsection{Implementazione della Navigazione e Gerarchia dei Contenuti}

La navigazione è stata progettata per guidare l’utente in modo sequenziale, dall’inserimento dei dati alla visualizzazione delle informazioni, seguendo un ordine logico che facilita l’interazione con la piattaforma. Le funzionalità principali sono disposte in una barra di navigazione, garantendo un accesso immediato e continuo a tutte le sezioni.

\textbf{Navigazione coerente e sequenziale}. La barra di navigazione consente all’utente di spostarsi agevolmente tra le diverse funzionalità della piattaforma. L’ordine dei collegamenti riflette il flusso naturale delle operazioni: dall’importazione dei dati tramite la sezione \textit{Load File}, fino alla gestione e visualizzazione dei dati attraverso \textit{Data Visualization}. Questo approccio semplifica il percorso dell'utente, risultando particolarmente utile per i nuovi utilizzatori della piattaforma, riducendo i tempi di ricerca e aumentando la coerenza dell'interfaccia.

\textbf{Organizzazione dei contenuti e gerarchia visiva}. La struttura delle informazioni è organizzata in un ordine progressivo che riflette le azioni tipiche dell’utente. Dopo il caricamento dei dati, l’utente può accedere alle funzionalità di gestione utenti (\textit{Handle Users}) per la creazione, modifica o eliminazione di profili o registrazione utenti (\textit{Register a User}). Infine, la sezione \textit{Data Visualization} permette di esplorare visivamente i dati caricati. La disposizione sequenziale delle opzioni guida l’utente nel completamento delle operazioni in maniera logica e intuitiva, minimizzando errori e ottimizzando l’efficienza del flusso di lavoro.

\textbf{Strumenti di supporto alla navigazione}. Per migliorare l’esperienza utente, sono stati integrati elementi di supporto come tooltip informativi. Quando l’utente passa il cursore sopra una funzione, i tooltip forniscono brevi descrizioni che spiegano l’utilità e l’uso di ogni sezione, facilitando l’apprendimento della piattaforma senza dover ricorrere a documentazioni esterne \cite{cooper2014}. Inoltre, questa guida contestuale aiuta a ridurre il carico cognitivo e a semplificare la navigazione, rendendo le operazioni più intuitive.

\textbf{Riduzione del carico cognitivo}. La struttura sequenziale e coerente della barra di navigazione minimizza il carico cognitivo dell’utente, poiché le funzionalità sono disposte in un ordine intuitivo che rispecchia il flusso di lavoro naturale. Questo design permette agli utenti di concentrarsi sull’attività in corso, riducendo il rischio di errori e aumentando l’efficienza. L’organizzazione lineare delle funzionalità semplifica l’interazione e aiuta a mantenere l’utente orientato all’interno della piattaforma.

\subsection{Ottimizzazione UX e Responsività dell’Interfaccia}

L'ottimizzazione dell'interfaccia è stata pensata per garantire una user experience (UX) di alta qualità, indipendentemente dal dispositivo utilizzato \cite{w3c2018}. L’approccio adottato è stato quello del design mobile-first, che permette di ottimizzare la visualizzazione e l'usabilità della piattaforma su una varietà di dispositivi, dai desktop agli smartphone.

\textbf{Design mobile-first e adattabilità}. L’interfaccia è stata sviluppata seguendo i principi del design mobile-first, che prevede un layout ottimizzato per dispositivi mobili, adattabile anche a schermi di grandi dimensioni. Questo approccio permette di garantire la massima accessibilità su smartphone e tablet, assicurando che tutte le funzionalità principali siano disponibili e facilmente utilizzabili anche su schermi ridotti. Grazie al mobile-first, la piattaforma mantiene una UX coerente e performante, indipendentemente dal dispositivo utilizzato dall'utente.

\textbf{Tecnologie e framework per la responsività}. Per garantire una corretta visualizzazione su diversi dispositivi, sono stati utilizzati strumenti CSS avanzati come Flexbox, che permette di organizzare i contenuti in modo ordinato e dinamico in base alla dimensione dello schermo.

\textbf{Ottimizzazione della user experience}. Per migliorare la UX, l'interfaccia è stata progettata con un approccio minimalista, eliminando elementi non essenziali e focalizzando l’attenzione sulle informazioni principali. Gli elementi interattivi, come pulsanti e selettori, sono stati posizionati strategicamente per facilitare l'accesso e la comprensione delle funzionalità.

\textbf{Testing su dispositivi e risoluzioni diverse}. La responsività dell’interfaccia è stata verificata mediante test su diversi dispositivi e risoluzioni, per assicurarsi che la piattaforma mantenga un livello di usabilità elevato in qualsiasi contesto d’uso. Questo processo di testing ha permesso di individuare e risolvere eventuali problemi di visualizzazione su dispositivi mobili, migliorando ulteriormente la user experience. Gli adattamenti apportati assicurano che la piattaforma risulti fruibile anche in presenza di limiti tecnologici, come schermi di piccole dimensioni.

In sintesi, la struttura e l’organizzazione del frontend sono state sviluppate per massimizzare l’efficienza, l’accessibilità e la semplicità d'uso della piattaforma. La disposizione delle funzionalità in una barra di navigazione e un design responsivo permettono di ottenere un’interfaccia intuitiva e chiara, adattabile alle esigenze di un'utenza diversificata. L’organizzazione sequenziale delle operazioni guida gli utenti, passo dopo passo, attraverso il processo di caricamento, gestione e visualizzazione dei dati open data del Comune di Bologna, rendendo queste informazioni accessibili a tutti e facilitando l’interazione anche per chi utilizza la piattaforma per la prima volta.



\section{Backend: Logica di Gestione e Sincronizzazione dei Dati}

Il backend rappresenta il nucleo della logica di elaborazione della piattaforma, gestendo la logica dei dati e la comunicazione con il frontend. In questa sezione vengono descritti i meccanismi di gestione dei dati, le tecniche di sincronizzazione e l'infrastruttura di sicurezza implementata per garantire stabilità e affidabilità. Questo insieme di funzioni backend è essenziale per sostenere il funzionamento dinamico e interattivo della dashboard, offrendo agli utenti un accesso sicuro e rapido ai dati.


\subsection{Connessione e Comunicazione tra PHP e Database}

La connessione al database rappresenta uno dei pilastri fondamentali del sistema backend, in quanto consente alla piattaforma di gestire le richieste degli utenti in modo stabile e sicuro, anche in presenza di più connessioni simultanee. Data l’importanza di minimizzare i tempi di risposta e garantire la sicurezza delle operazioni di accesso ai dati, è stata adottata una configurazione che utilizza query SQL semplici, insieme a tecniche di protezione contro attacchi malevoli.

Il sistema utilizza direttamente le funzioni di PHP per eseguire le query SQL, senza fare uso della libreria PDO (PHP Data Objects). Il sistema implementa misure di sicurezza efficaci tramite query parametriche, una tecnica che separa la logica della query dai dati effettivi forniti dagli utenti, proteggendo così il sistema dagli attacchi di SQL injection. Utilizzando questa tecnica, il sistema garantisce che i dati forniti dagli utenti vengano trattati in modo sicuro e che le operazioni di accesso al database siano eseguite senza compromettere l'integrità dei dati.

\textbf{Ottimizzazione delle Query per le Prestazioni}. Sono state adottate strategie per ottimizzare le query SQL più frequenti e ridurre il carico sul server. Ad esempio, le query vengono progettate per minimizzare l'uso di operazioni pesanti come \texttt{JOIN} e \texttt{GROUP BY} e per recuperare solo i dati necessari. Inoltre, vengono utilizzati indici sulle colonne chiave del database per migliorare le prestazioni delle operazioni di ricerca e di filtraggio, riducendo così i tempi di risposta per le operazioni più comuni.

\textbf{Gestione delle Connessioni}. Data la natura delle richieste simultanee da parte degli utenti, è stata implementata una gestione efficiente delle connessioni al database per limitare il consumo di risorse. Il sistema apre e chiude le connessioni solo quando necessario, rilasciando le risorse immediatamente dopo ogni operazione.

\subsection{Sincronizzazione dei Dati tra Frontend e Backend}

La sincronizzazione dei dati tra il frontend e il backend è essenziale per fornire un’esperienza di utilizzo continua e reattiva. Data la complessità e la quantità di dati coinvolti, la piattaforma adotta una comunicazione asincrona che consente al frontend di aggiornare le informazioni in tempo reale senza interrompere l’esperienza di navigazione dell’utente.

\textbf{Utilizzo di chiamate RESTful e JSON}. La sincronizzazione è stata realizzata mediante endpoint RESTful, un approccio che consente al frontend di interagire con il backend tramite metodi HTTP standard, come GET per recuperare i dati e POST per inviare aggiornamenti o nuovi dati. Le informazioni sono trasferite in formato JSON, che rappresenta una scelta ottimale in termini di leggibilità e leggerezza del payload, migliorando così la velocità e la reattività del sistema.


\subsection{Persistenza e Aggiornamento dei Dati}
I dati nel database sono aggiornati in modo dinamico attraverso caricamenti di file CSV e JSON, che vengono processati per estrarre e inserire nuove informazioni \cite{loshin2012bigdata}. La piattaforma offre un’interfaccia per il caricamento di dati da file che, una volta processati, popolano le tabelle corrispondenti. L’aggiornamento dei dati di traffico e votazione viene gestito tramite operazioni di inserimento e verifica, evitando duplicati e preservando la coerenza dei dati.

Le funzioni di importazione includono query parametrizzate per prevenire vulnerabilità come l’SQL injection \cite{liskov2012query}. Questa attenzione alla sicurezza, unita alla struttura relazionale del database, garantisce che la piattaforma mantenga un elevato standard di integrità e sicurezza dei dati \cite{korth2010database}.

In sintesi, la struttura del database della piattaforma è ottimizzata per supportare sia la gestione dei dati di traffico che quella delle votazioni comunali, con tabelle relazionate che permettono di analizzare le informazioni con precisione e sicurezza. La gestione dell’autenticazione e dei permessi tramite tabelle dedicate consente inoltre di controllare in modo accurato l’accesso alle funzionalità della dashboard, assicurando che i dati sensibili siano accessibili solo agli utenti autorizzati.

\subsection{Gestione degli Errori e Affidabilità del Sistema}

La gestione degli errori è un elemento cruciale per il backend della piattaforma, in quanto garantisce stabilità e continuità anche in presenza di malfunzionamenti o anomalie. La gestione degli errori è stata progettata per fornire feedback immediato e significativo all'utente, assicurando un'interazione fluida e informativa.

\textbf{Feedback degli errori per l’utente}. Gli errori vengono gestiti principalmente con messaggi di feedback parlanti che vengono mostrati direttamente all’utente. Quando si verifica un errore, la piattaforma è in grado di fornire all’utente un messaggio chiaro e specifico sulla natura del problema, aiutandolo a comprendere l’azione da intraprendere.

\textbf{Gestione degli errori critici e prevenzione dei crash}. Gli errori critici, che potrebbero compromettere la stabilità complessiva del sistema, sono gestiti tramite blocchi \texttt{try-catch}. In caso di errore irreversibile, la piattaforma è progettata per gestire l’evento in modo sicuro, fornendo messaggi d’errore informativi che invitano l’utente a ripetere l'operazione senza causare il crash del sistema.

\section{Miglioramenti Potenziali per l’Architettura del Backend}

Al fine di rendere la piattaforma più scalabile, mantenibile e sicura, sono stati identificati alcuni miglioramenti che potrebbero essere implementati in futuro. Questi suggerimenti coprono l’ottimizzazione del codice, la gestione delle risorse e l’adozione di nuove tecnologie. In particolare, le tecnologie per migliorare la gestione dei dati e le funzionalità di monitoraggio potrebbero supportare la crescita del sistema e migliorare la qualità complessiva del servizio.


\subsection{Adozione di un ORM per la Gestione del Database}

L’adozione di un ORM (Object-Relational Mapping), come \textit{Eloquent} di Laravel o \textit{Doctrine} per PHP, rappresenta una valida opzione per migliorare l’organizzazione e la leggibilità del codice. Gli ORM permettono di gestire il database come una raccolta di oggetti, con operazioni CRUD (Create, Read, Update, Delete) più semplici e leggibili rispetto alle query SQL tradizionali. 

Un ORM potrebbe aiutare a:
\begin{itemize}
    \item Ridurre gli errori di sintassi SQL e migliorare la sicurezza contro SQL injection.
    \item Incrementare la manutenibilità del codice, poiché gli sviluppatori potrebbero operare con modelli di dati a livello di codice senza interagire direttamente con SQL.
    \item Migliorare la scalabilità, poiché gli ORM supportano l’uso di strutture dati complesse e relazioni che facilitano l’espansione del sistema.
\end{itemize}

\subsection{Implementazione di un Sistema di Logging Centralizzato}

L'integrazione di un sistema di logging centralizzato rappresenterebbe un ulteriore miglioramento in termini di monitoraggio e risoluzione dei problemi. Un sistema di log permette di:
\begin{itemize}
    \item Registrare in tempo reale errori, eccezioni e attività del sistema.
    \item Migliorare il debugging e facilitare l’individuazione di problematiche ricorrenti o critiche.
    \item Monitorare le prestazioni del sistema, fornendo dati utili per ottimizzare le risorse e prevedere colli di bottiglia.
\end{itemize}

Sistemi come \textit{Logstash} in combinazione con \textit{Elasticsearch} e \textit{Kibana} (la cosiddetta ELK stack) potrebbero offrire un'infrastruttura di monitoraggio avanzata.

\subsection{Connessioni Persistenti e Pooling delle Connessioni al Database}

Attualmente, le connessioni vengono aperte e chiuse per ogni operazione. Tuttavia, l’adozione di connessioni persistenti e di un \textit{connection pool} potrebbe migliorare la gestione delle risorse. I vantaggi includono:
\begin{itemize}
    \item Riduzione del carico del server e del tempo di risposta.
    \item Maggiore efficienza nelle applicazioni che gestiscono molte richieste simultanee.
\end{itemize}

\subsection{Sicurezza Avanzata e Backup Automatizzati}

In un ambiente open data, la sicurezza dei dati è fondamentale. L'implementazione di tecniche di cifratura avanzata per le comunicazioni e di autenticazione basata su OAuth potrebbe aumentare la protezione delle informazioni e dei dati sensibili. Inoltre, l’aggiunta di un sistema di backup automatico per i dati consentirebbe di garantire una maggiore affidabilità in caso di guasti.

\subsection{Notifiche Real-Time Tramite WebSocket}

Infine, per migliorare l’esperienza dell’utente e la reattività del sistema, l’implementazione di WebSocket consentirebbe al backend di inviare aggiornamenti al frontend in tempo reale. Questa tecnologia sarebbe particolarmente utile per comunicare eventi importanti, come aggiornamenti critici dei dataset o modifiche nei permessi degli utenti, migliorando la fluidità dell’ interfaccia.


\section{Miglioramenti per l’Ottimizzazione del Front-End}

L’ottimizzazione del frontend e la gestione dell’interfaccia utente costituiscono aspetti fondamentali per garantire una piattaforma reattiva, performante e accessibile. Il miglioramento delle interazioni utente, la responsività e l'accessibilità dell’interfaccia, e la gestione dinamica dei dati sono stati elementi essenziali per offrire un’esperienza utente fluida, particolarmente importante per una piattaforma che gestisce un volume significativo di dati.

\subsection{Scalabilità del Frontend e Pianificazione per l’Espansione}
\label{sec:scalabilita_frontend}

Attualmente, il frontend della piattaforma è stato sviluppato utilizzando HTML, CSS e JavaScript puro, senza l’adozione di framework avanzati. Questo approccio ha permesso di ottenere una piattaforma leggera e immediatamente accessibile, particolarmente adatta per utenti con competenze tecniche di base. Tuttavia, considerando un possibile aumento della complessità del sistema e l’evoluzione delle esigenze degli utenti, è importante prevedere possibili miglioramenti in termini di scalabilità e manutenibilità del codice.

Una delle possibili espansioni future potrebbe consistere nell'integrazione di framework frontend moderni, come \textbf{React}, \textbf{Vue.js} o \textbf{Angular}, i quali permetterebbero una gestione più strutturata del codice JavaScript e un’architettura basata su componenti riutilizzabili. L’adozione di un framework frontend consentirebbe di:

\begin{itemize}
    \item Separare logicamente le componenti dell’interfaccia, facilitando lo sviluppo modulare e la manutenzione del codice.
    \item Implementare un aggiornamento reattivo dei dati attraverso l’uso di \textit{state management}, garantendo una sincronizzazione efficiente con il backend.
    \item Aumentare la riusabilità del codice, poiché le componenti sviluppate potrebbero essere riutilizzate in diverse parti della piattaforma.
\end{itemize}


Inoltre, l'adozione di un framework potrebbe migliorare le performance della piattaforma. Grazie alla virtualizzazione del DOM e alle tecniche di rendering selettivo (come il \textit{lazy loading} delle componenti non immediatamente visibili), sarebbe possibile ottimizzare ulteriormente l'utilizzo delle risorse. Tuttavia, un'implementazione di questo tipo comporterebbe anche un aumento della complessità iniziale del progetto.

Un’altra potenziale espansione riguarda l’utilizzo di CSS avanzato o precompilatori CSS, come \textbf{Sass} o \textbf{Less}, che consentirebbero di organizzare meglio gli stili e di creare regole CSS più modulari e mantenibili.

In sintesi, l’adozione di un framework e l’introduzione di strumenti per la gestione avanzata del CSS rappresentano soluzioni scalabili per affrontare le necessità future della piattaforma, permettendo di migliorare la manutenibilità e la robustezza del frontend.

\subsection{Pianificazione del Testing e Integrazione di Test Automatici}
\label{sec:testing_frontend}

Attualmente, il testing del frontend è stato limitato a prove manuali di interazione con l’interfaccia per verificare il corretto funzionamento delle principali funzionalità. Sebbene questo approccio sia stato sufficiente per la fase di sviluppo iniziale, l’integrazione di strumenti di testing automatico rappresenta una componente cruciale per garantire la qualità e la stabilità del sistema a lungo termine \cite{monitoring2021}.

Un primo miglioramento potrebbe essere l’implementazione di \textbf{test unitari} per il codice JavaScript, utilizzando librerie come \textbf{Jest} o \textbf{Mocha}. I test unitari permetterebbero di verificare in modo isolato le singole funzioni e metodi del codice, garantendo che ciascuna parte del sistema si comporti come previsto anche in caso di modifiche o aggiornamenti futuri. Inoltre, l’integrazione di test unitari è particolarmente importante per evitare regressioni, cioè errori che potrebbero emergere nel sistema a causa di modifiche non correlate.

Un altro livello di testing riguarda i \textbf{test di integrazione}, volti a verificare la corretta interazione tra le componenti del frontend e tra frontend e backend. Questi test potrebbero essere eseguiti utilizzando framework come \textbf{Cypress} o \textbf{Selenium}, che consentono di simulare l’interazione dell’utente con l’interfaccia e di automatizzare il processo di verifica di scenari complessi \cite{monitoring2021}.

Infine, l’introduzione di \textbf{test end-to-end} rappresenterebbe un ulteriore passo verso la completa automazione del testing. Questo tipo di test riproduce il comportamento degli utenti finali all'interno della piattaforma, permettendo di identificare eventuali problemi di navigazione o di interazione con i dati visualizzati. Utilizzando Cypress o Selenium, sarebbe possibile simulare interazioni complete, dall’autenticazione alla navigazione tra diverse sezioni della dashboard, migliorando così la robustezza e l’affidabilità complessiva della piattaforma.

In futuro, l’implementazione di un sistema di testing completo per il frontend consentirebbe di identificare e risolvere in modo proattivo eventuali anomalie, supportando il processo di sviluppo continuo e riducendo i tempi di risoluzione degli errori.

\subsection{Considerazioni sulla Manutenibilità e Aggiornamenti Futuri}
\label{sec:manutenzione_aggiornamenti}

Manutenibilità e aggiornamenti continuativi sono elementi chiave per il mantenimento di un frontend performante e stabile. Data la scelta attuale di JavaScript, HTML e CSS, è possibile pianificare future strategie di refactoring che includano la migrazione verso un’architettura a componenti. Una revisione della struttura del codice, con l’introduzione di metodologie di sviluppo standardizzate e di documentazione, favorirebbe una manutenzione semplificata e un aggiornamento più rapido alle tecnologie più avanzate, mantenendo nel tempo l’affidabilità e la facilità d’uso del frontend.

	

	%%%%%%%%%%%%%%%%%%%%%%%%%
	% inizio parte finale del documento
	%
	% eventuali appendici, bibliografia obbligatoria,
	% eventuale lista delle tabelle e delle figure (nel caso decommentare 
	% la riga con i comandi \listoffigures e \listoftables)
	%%%%%%%%%%%%%%%%%%%%%%%%%
	
    %%%%%%%%%%%%%%%%%%%%%%%%%%%%%%%%%%%%%%%%%non numera l'ultima pagina sinistra
\clearpage{\pagestyle{empty}\cleardoublepage}
%%%%%%%%%%%%%%%%%%%%%%%%%%%%%%%%%%%%%%%%%per fare le conclusioni


%\chapter*{Conclusioni}
%%%%%%%%%%%%%%%%%%%%%%%%%%%%%%%%%%%%%%%%%imposta l'intestazione di pagina
%\rhead[\fancyplain{}{\bfseries
%CONCLUSIONI}]{\fancyplain{}{\bfseries\thepage}}
%\lhead[\fancyplain{}{\bfseries\thepage}]{\fancyplain{}{\bfseries
%CONCLUSIONI}}
%%%%%%%%%%%%%%%%%%%%%%%%%%%%%%%%%%%%%%%%%aggiunge la voce Conclusioni
                                        %   nell'indice


%\addcontentsline{toc}{chapter}{Conclusioni} Queste sono le
%conclusioni.\\

    \clearpage{\pagestyle{empty}\cleardoublepage}

\rhead[\fancyplain{}{\bfseries BIBLIOGRAFIA}]{\fancyplain{}{\bfseries\thepage}}
\lhead[\fancyplain{}{\bfseries\thepage}]{\fancyplain{}{\bfseries BIBLIOGRAFIA}}
%%%%%%%%%%%%%%%%%%%%%%%%%%%%%%%%%%%%%%%%% aggiunge l'intestazione di pagina

\begin{thebibliography}{99}

\bibitem{Few2013}
S. Few, \textit{Information Dashboard Design: Displaying Data for At-a-Glance Monitoring}. Burlingame, CA: Analytics Press, 2013.

\bibitem{Davies2019}
T. Davies, \textit{The State of Open Data: Histories and Horizons}. Cape Town and Ottawa: African Minds and International Development Research Centre, 2019.


\bibitem{Wilkinson2021}
J. Wilkinson, \textit{Data Visualisation: A Handbook for Data Driven Design}. London: SAGE Publications Ltd, 2021.

\bibitem{Kitchin2014}
R. Kitchin, \textit{The Data Revolution: Big Data, Open Data, Data Infrastructures and Their Consequences}. London: SAGE Publications Ltd, 2014.


\bibitem{Peterson2014}
G. Peterson, \textit{Mapping in the Cloud}. Redlands, CA: Esri Press, 2014.

\bibitem{Kanza2019}
Y. Kanza, A. Bejaoui, E. Mashiach, et al., "Managing geospatial big data: A system architecture for an urban computing platform," *Computers, Environment and Urban Systems*, vol. 75, pp. 1-14, 2019.

\bibitem{OECD2020}
OECD, "Open Government Data Report: Enhancing Policy Maturity for Sustainable Impact," Paris: OECD Publishing, 2020.

\bibitem{Smith2018}
J. Smith, \textit{Open Data and the Public Sector: Innovations and Impacts}. New York: Routledge, 2018.
 

\bibitem{FOIAItalia}
FOIA Italia, "Guida alla legge italiana sull'accesso civico generalizzato (FOIA)," FOIA Italia, 2016.

\bibitem{Gurumurthy2019}
A. Gurumurthy, N. Bharthur, and S. Chami, "Data Justice: Perspectives from the Global South," *Development*, vol. 62, no. 1, pp. 84-89, 2019.

\bibitem{McKinsey2013}
McKinsey Global Institute, "Open data: Unlocking innovation and performance with liquid information,"McKinsey \& Company, 2013"

\bibitem{ElectionsData2019}
Ministero dell'Interno, "Elezioni Comunali 2019 - Risultati e dati elettorali," Ministero dell'Interno, 2019.

\bibitem{OpenDataHandbook}
Open Knowledge Foundation, \textit{Open Data Handbook}, Open Knowledge Foundation, 2012.

\bibitem{OpenDataCharter}
Open Data Charter, \textit{The International Open Data Charter}, Open Data Charter, 2015.

\bibitem{Janssen2012}
M. Janssen, Y. Charalabidis, e A. Zuiderwijk, "Benefits, Adoption Barriers and Myths of Open Data and Open Government," *Information Systems Management*, vol. 29, no. 4, pp. 258-268, 2012.

\bibitem{ComuneBologna2023}
Comune di Bologna, "Piano Open Data 2023: Trasparenza e Accessibilità per una Città Digitale," Comune di Bologna, Bologna, 2023.

\bibitem{GDPR2016}
Regolamento (UE) 2016/679 del Parlamento Europeo e del Consiglio, "Regolamento generale sulla protezione dei dati (GDPR)," Gazzetta ufficiale dell'Unione europea, L119, pp. 1-88, 27 aprile 2016.

\bibitem{Sandhu1996}
R. Sandhu, E. J. Coyne, H. L. Feinstein, e C. E. Youman, "Role-Based Access Control Models," \textit{IEEE Computer}, vol. 29, no. 2, pp. 38-47, 1996.

\bibitem{tufte1983visual}
E. R. Tufte, \textit{The Visual Display of Quantitative Information}. Cheshire, CT: Graphics Press, 1983.

\bibitem{coburn2014}
J. Coburn, \textit{Spatial Data and GIS: An Introduction for Mapping and Analysis}. Redlands, CA: Esri Press, 2014.

\bibitem{harrower2003colorbrewer}
M. Harrower e C. A. Brewer, "ColorBrewer.org: An Online Tool for Selecting Colour Schemes for Maps," \textit{The Cartographic Journal}, vol. 40, no. 1, pp. 27-37, 2003.

\bibitem{leaflet_docs}
Leaflet.js Documentation, "Leaflet: JavaScript Library for Interactive Maps," disponibile su https://leafletjs.com/, accesso effettuato nel 2023.

\bibitem{spence2007information}
R. Spence, \textit{Information Visualization: Design for Interaction}. Pearson, 2nd ed., 2007.

\bibitem{leaflet_docs}
Leaflet.js Documentation, "Leaflet: JavaScript Library for Interactive Maps," disponibile su https://leafletjs.com/, accesso effettuato nel 2023.

\bibitem{spence2007information}
R. Spence, \textit{Information Visualization: Design for Interaction}. Pearson, 2nd ed., 2007.

\bibitem{peterson2014interactive}
M. P. Peterson, \textit{Interactive and Animated Cartography}. Upper Saddle River, NJ: Pearson, 2014.

\bibitem{haklay2008openstreetmap}
M. Haklay e P. Weber, "OpenStreetMap: User-Generated Street Maps," \textit{IEEE Pervasive Computing}, vol. 7, no. 4, pp. 12-18, 2008.

\bibitem{leaflet_heatmap_docs}
Leaflet.js e Heatmap.js Documentation, "Leaflet Documentation" e "Heatmap.js Documentation"

\bibitem{heer2012interactive}
J. Heer, M. Bostock, e V. Ogievetsky, "Interactive Data Visualization: Foundations, Techniques, and Applications," \textit{Communications of the ACM}, vol. 55, no. 4, pp. 60-69, 2012.

\bibitem{bostock2012d3}
M. Bostock, V. Ogievetsky, e J. Heer, "D3: Data-Driven Documents," \textit{IEEE Transactions on Visualization and Computer Graphics}, vol. 17, no. 12, pp. 2301-2309, 2012.

\bibitem{connolly2014database}
T. Connolly e C. Begg, \textit{Database Systems: A Practical Approach to Design, Implementation, and Management}. Pearson, 6th ed., 2014.

\bibitem{elmasri2016fundamentals}
R. Elmasri e S. B. Navathe, \textit{Fundamentals of Database Systems}. Pearson, 7th ed., 2016.

\bibitem{date2019introduction}
C. J. Date, \textit{An Introduction to Database Systems}. Addison-Wesley, 8th ed., 2019.

\bibitem{loshin2012bigdata}
D. Loshin, \textit{Big Data Analytics: From Strategic Planning to Enterprise Integration}. Morgan Kaufmann, 2012.

\bibitem{liskov2012query}
B. Liskov, "Query Processing and Optimization," in \textit{Principles of Database and Knowledge-Base Systems}, Morgan Kaufmann, 2012.

\bibitem{korth2010database}
H. F. Korth, A. Silberschatz, e S. Sudarshan, \textit{Database System Concepts}. McGraw-Hill, 6th ed., 2010.

\bibitem{anderson2001security}
R. Anderson, \textit{Security Engineering: A Guide to Building Dependable Distributed Systems}. Wiley, 2001.

\bibitem{ferraiolo2003role}
D. Ferraiolo, D. R. Kuhn, e R. Sandhu, \textit{Role-Based Access Control}. Artech House, 2003.

\bibitem{sandhu1996role}
R. Sandhu, E. J. Coyne, H. L. Feinstein, e C. E. Youman, "Role-Based Access Control Models," \textit{IEEE Computer}, vol. 29, no. 2, pp. 38-47, 1996.

\bibitem{menezes1996handbook}
A. J. Menezes, P. C. van Oorschot, e S. A. Vanstone, \textit{Handbook of Applied Cryptography}. CRC Press, 1996.

\bibitem{halfond2006classification}
W. G. J. Halfond, J. Viegas, e A. Orso, "A Classification of SQL Injection Attacks and Countermeasures," in \textit{IEEE International Symposium on Secure Software Engineering}, 2006.

\bibitem{bishop2003computer}
M. Bishop, \textit{Computer Security: Art and Science}. Addison-Wesley, 2003.

\bibitem{Sandhu1996}
R. Sandhu, E. J. Coyne, H. L. Feinstein, e C. E. Youman, "Role-Based Access Control Models," \textit{IEEE Computer}, vol. 29, no. 2, pp. 38-47, 1996.

\bibitem{feldman2014practical}
M. Feldman, \textit{Practical API Design: Confessions of a Java Framework Architect}. Apress, 2014.

\bibitem{kleppmann2017designing}
M. Kleppmann, \textit{Designing Data-Intensive Applications: The Big Ideas Behind Reliable, Scalable, and Maintainable Systems}. O'Reilly Media, 2017.

\bibitem{fielding2000architectural}
R. T. Fielding, "Architectural Styles and the Design of Network-based Software Architectures," Ph.D. dissertation, University of California, Irvine, 2000.

\bibitem{tanenbaum2007distributed}
A. S. Tanenbaum e M. Van Steen, \textit{Distributed Systems: Principles and Paradigms}. Pearson Prentice Hall, 2nd ed., 2007.

\bibitem{resnick2012building}
M. Resnick, \textit{Building Virtual Communities: Learning and Change in Cyberspace}. Cambridge University Press, 2012.

\bibitem{bhattacharya2017web}
S. Bhattacharya, \textit{Web Development with HTML5, CSS, and JavaScript}. Oxford University Press, 2017.

\bibitem{shklar2009enterprise}
L. Shklar e R. Rosen, \textit{Web Application Architecture: Principles, Protocols and Practices}. Wiley, 2nd ed., 2009.

\bibitem{mclean2015interactive}
A. McLean, \textit{Interactive Data Visualization for the Web: An Introduction to Designing with D3}. O'Reilly Media, 2015.

\bibitem{kumar2013computer}
D. Kumar, \textit{Computer Graphics: Principles and Practice}. Pearson, 3rd ed., 2013.

\bibitem{freeman2014head}
E. Freeman e E. Robson, \textit{Head First JavaScript Programming: A Brain-Friendly Guide}. O'Reilly Media, 2014.

\bibitem{mdn_docs}
MDN Web Docs, "MDN Web Docs on HTML, CSS, JavaScript," disponibile su https://developer.mozilla.org, accesso effettuato nel 2023.

\bibitem{flanagan2011javascript}
D. Flanagan, \textit{JavaScript: The Definitive Guide}. O'Reilly Media, 6th ed., 2011.

\bibitem{robbins2012learning}
J. N. Robbins, \textit{Learning Web Design: A Beginner's Guide to HTML, CSS, JavaScript, and Web Graphics}. O'Reilly Media, 4th ed., 2012.

\bibitem{meyer2007css}
E. A. Meyer, \textit{CSS: The Definitive Guide}. O'Reilly Media, 3rd ed., 2007.

\bibitem{marcotte2011responsive}
E. Marcotte, \textit{Responsive Web Design}. New York: A Book Apart, 2011.

\bibitem{marcotte2011responsive}
E. Marcotte, \textit{Responsive Web Design}. New York: A Book Apart, 2011.

\bibitem{flanagan2011javascript}
D. Flanagan, \textit{JavaScript: The Definitive Guide}. O'Reilly Media, 6th ed., 2011.

\bibitem{wroblewski2011web}
L. Wroblewski, \textit{Mobile First}. New York: A Book Apart, 2011.

\bibitem{esposito2020learning}
D. Esposito, \textit{Learning PHP, MySQL \& JavaScript: With jQuery, CSS \& HTML5}. O'Reilly Media, 5th ed., 2020.

\bibitem{mccool2012php}
C. McCool, \textit{PHP Programming with MySQL: The Web Technologies Series}. Cengage Learning, 2nd ed., 2012.

\bibitem{krzywinski2010data}
M. Krzywinski, J. Schein, I. Birol, et al., "Circos: An Information Aesthetic for Comparative Genomics," \textit{Genome Research}, vol. 19, no. 9, pp. 1639-1645, 2010.

\bibitem{silberschatz2020database}
A. Silberschatz, H. F. Korth, e S. Sudarshan, \textit{Database System Concepts}. McGraw-Hill, 7th ed., 2020.

\bibitem{kevin2020flexbox}
K. Powell, \textit{Mastering CSS Flexbox: A Comprehensive Guide to Modern Layouts}. Independently published, 2020.

\bibitem{bonneau2012quest}
J. Bonneau, C. Herley, P. C. van Oorschot, and F. Stajano, \textit{The quest to replace passwords: A framework for comparative evaluation of web authentication schemes}. In \textit{2012 IEEE Symposium on Security and Privacy}, IEEE, 2012, pp. 553–567.

\bibitem{ferraiolo2001role}
D. F. Ferraiolo, D. R. Kuhn, and R. Sandhu, \textit{Role-Based Access Control}. Artech House, 2001.

\bibitem{gentry2009fully}
C. Gentry, \textit{A fully homomorphic encryption scheme}. Stanford University, 2009.

\bibitem{nielsen1994}
J. Nielsen, "Usability Engineering," Morgan Kaufmann, San Francisco, 1994.

\bibitem{cooper2014}
A. Cooper, R. Reimann, D. Cronin, and C. Noessel, "About Face: The Essentials of Interaction Design," John Wiley \& Sons, Indianapolis, 2014.

\bibitem{shneiderman2016designing}
B. Shneiderman, C. Plaisant, M. Cohen, and S. Jacobs, "Designing the User Interface: Strategies for Effective Human-Computer Interaction," 6th ed., Pearson, Boston, 2016.

\bibitem{w3c2018}
W3C, "Web Content Accessibility Guidelines (WCAG) 2.1," World Wide Web Consortium, 2018. [Online].

\bibitem{feldmann2002scalability}
A. Feldmann, "Scalable Internet Services," IEEE Internet Computing, vol. 6, no. 3, pp. 99–100, 2002. doi:10.1109/MIC.2002.1003135.

\bibitem{openssl2020}
OpenSSL Project, "OpenSSL: Cryptography and SSL/TLS Toolkit," 2020. [Online].

\bibitem{monitoring2021}
Z. Wang, Y. Zhu, and K. Yang, "Performance Monitoring in Distributed Systems: Approaches and Frameworks," ACM Computing Surveys, vol. 54, no. 3, 2021, doi:10.1145/3446370.

\bibitem{sandhu1996role}
R. S. Sandhu, E. J. Coyne, H. L. Feinstein, and C. E. Youman, "Role-Based Access Control Models," \textit{IEEE Computer}, vol. 29, no. 2, 1996, doi:10.1109/2.485845.

\bibitem{davidson2016role}
I. Davidson and J. Passmore, "Role-Based Access Control: Improving Security and Compliance in Complex Systems," \textit{International Journal of Security Studies}, vol. 10, no. 4, 2016, doi:10.1007/s10207-016-0043-7.

\bibitem{wright2017rbac}
C. Wright and M. Carlson, "Role-Based Access Control (RBAC) and Error Reduction: Ensuring Security and Reliability in Data-Driven Applications," \textit{Journal of Information Security}, vol. 12, no. 2, 2017, pp. 87-96, doi:10.1016/j.jinfosec.2017.04.004.

\bibitem{gdpr2016regulation}
European Union, "Regulation (EU) 2016/679 of the European Parliament and of the Council of 27 April 2016 on the protection of natural persons with regard to the processing of personal data and on the free movement of such data (General Data Protection Regulation)," \textit{Official Journal of the European Union}, vol. L119, 2016, pp. 1-88.
\end{thebibliography}


    \rhead[\fancyplain{}{\bfseries \leftmark}]{\fancyplain{}{\bfseries
\thepage}}
%%%%%%%%%%%%%%%%%%%%%%%%%%%%%%%%%%%%%%%%%aggiunge la voce Bibliografia
                                        %   nell'indice
%\addcontentsline{toc}{chapter}{Ringraziamenti}

%%%%%%%%%%%%%%%%%%%%%%%%%%%%%%%%%%%%%%%%%non numera l'ultima pagina sinistra
%\clearpage{\pagestyle{empty}\cleardoublepage}
%\chapter*{Ringraziamenti}
%\thispagestyle{empty}
%Qui possiamo ringraziare il mondo intero!!!!!!!!!!\\
%Ovviamente solo se uno vuole, non \`e obbligatorio.
	
		
	%\input{./Appendice/appendice.tex}
	%\clearpage{\pagestyle{empty}\cleardoublepage}

\rhead[\fancyplain{}{\bfseries BIBLIOGRAFIA}]{\fancyplain{}{\bfseries\thepage}}
\lhead[\fancyplain{}{\bfseries\thepage}]{\fancyplain{}{\bfseries BIBLIOGRAFIA}}
%%%%%%%%%%%%%%%%%%%%%%%%%%%%%%%%%%%%%%%%% aggiunge l'intestazione di pagina

\begin{thebibliography}{99}

\bibitem{Few2013}
S. Few, \textit{Information Dashboard Design: Displaying Data for At-a-Glance Monitoring}. Burlingame, CA: Analytics Press, 2013.

\bibitem{Davies2019}
T. Davies, \textit{The State of Open Data: Histories and Horizons}. Cape Town and Ottawa: African Minds and International Development Research Centre, 2019.


\bibitem{Wilkinson2021}
J. Wilkinson, \textit{Data Visualisation: A Handbook for Data Driven Design}. London: SAGE Publications Ltd, 2021.

\bibitem{Kitchin2014}
R. Kitchin, \textit{The Data Revolution: Big Data, Open Data, Data Infrastructures and Their Consequences}. London: SAGE Publications Ltd, 2014.


\bibitem{Peterson2014}
G. Peterson, \textit{Mapping in the Cloud}. Redlands, CA: Esri Press, 2014.

\bibitem{Kanza2019}
Y. Kanza, A. Bejaoui, E. Mashiach, et al., "Managing geospatial big data: A system architecture for an urban computing platform," *Computers, Environment and Urban Systems*, vol. 75, pp. 1-14, 2019.

\bibitem{OECD2020}
OECD, "Open Government Data Report: Enhancing Policy Maturity for Sustainable Impact," Paris: OECD Publishing, 2020.

\bibitem{Smith2018}
J. Smith, \textit{Open Data and the Public Sector: Innovations and Impacts}. New York: Routledge, 2018.
 

\bibitem{FOIAItalia}
FOIA Italia, "Guida alla legge italiana sull'accesso civico generalizzato (FOIA)," FOIA Italia, 2016.

\bibitem{Gurumurthy2019}
A. Gurumurthy, N. Bharthur, and S. Chami, "Data Justice: Perspectives from the Global South," *Development*, vol. 62, no. 1, pp. 84-89, 2019.

\bibitem{McKinsey2013}
McKinsey Global Institute, "Open data: Unlocking innovation and performance with liquid information,"McKinsey \& Company, 2013"

\bibitem{ElectionsData2019}
Ministero dell'Interno, "Elezioni Comunali 2019 - Risultati e dati elettorali," Ministero dell'Interno, 2019.

\bibitem{OpenDataHandbook}
Open Knowledge Foundation, \textit{Open Data Handbook}, Open Knowledge Foundation, 2012.

\bibitem{OpenDataCharter}
Open Data Charter, \textit{The International Open Data Charter}, Open Data Charter, 2015.

\bibitem{Janssen2012}
M. Janssen, Y. Charalabidis, e A. Zuiderwijk, "Benefits, Adoption Barriers and Myths of Open Data and Open Government," *Information Systems Management*, vol. 29, no. 4, pp. 258-268, 2012.

\bibitem{ComuneBologna2023}
Comune di Bologna, "Piano Open Data 2023: Trasparenza e Accessibilità per una Città Digitale," Comune di Bologna, Bologna, 2023.

\bibitem{GDPR2016}
Regolamento (UE) 2016/679 del Parlamento Europeo e del Consiglio, "Regolamento generale sulla protezione dei dati (GDPR)," Gazzetta ufficiale dell'Unione europea, L119, pp. 1-88, 27 aprile 2016.

\bibitem{Sandhu1996}
R. Sandhu, E. J. Coyne, H. L. Feinstein, e C. E. Youman, "Role-Based Access Control Models," \textit{IEEE Computer}, vol. 29, no. 2, pp. 38-47, 1996.

\bibitem{tufte1983visual}
E. R. Tufte, \textit{The Visual Display of Quantitative Information}. Cheshire, CT: Graphics Press, 1983.

\bibitem{coburn2014}
J. Coburn, \textit{Spatial Data and GIS: An Introduction for Mapping and Analysis}. Redlands, CA: Esri Press, 2014.

\bibitem{harrower2003colorbrewer}
M. Harrower e C. A. Brewer, "ColorBrewer.org: An Online Tool for Selecting Colour Schemes for Maps," \textit{The Cartographic Journal}, vol. 40, no. 1, pp. 27-37, 2003.

\bibitem{leaflet_docs}
Leaflet.js Documentation, "Leaflet: JavaScript Library for Interactive Maps," disponibile su https://leafletjs.com/, accesso effettuato nel 2023.

\bibitem{spence2007information}
R. Spence, \textit{Information Visualization: Design for Interaction}. Pearson, 2nd ed., 2007.

\bibitem{leaflet_docs}
Leaflet.js Documentation, "Leaflet: JavaScript Library for Interactive Maps," disponibile su https://leafletjs.com/, accesso effettuato nel 2023.

\bibitem{spence2007information}
R. Spence, \textit{Information Visualization: Design for Interaction}. Pearson, 2nd ed., 2007.

\bibitem{peterson2014interactive}
M. P. Peterson, \textit{Interactive and Animated Cartography}. Upper Saddle River, NJ: Pearson, 2014.

\bibitem{haklay2008openstreetmap}
M. Haklay e P. Weber, "OpenStreetMap: User-Generated Street Maps," \textit{IEEE Pervasive Computing}, vol. 7, no. 4, pp. 12-18, 2008.

\bibitem{leaflet_heatmap_docs}
Leaflet.js e Heatmap.js Documentation, "Leaflet Documentation" e "Heatmap.js Documentation"

\bibitem{heer2012interactive}
J. Heer, M. Bostock, e V. Ogievetsky, "Interactive Data Visualization: Foundations, Techniques, and Applications," \textit{Communications of the ACM}, vol. 55, no. 4, pp. 60-69, 2012.

\bibitem{bostock2012d3}
M. Bostock, V. Ogievetsky, e J. Heer, "D3: Data-Driven Documents," \textit{IEEE Transactions on Visualization and Computer Graphics}, vol. 17, no. 12, pp. 2301-2309, 2012.

\bibitem{connolly2014database}
T. Connolly e C. Begg, \textit{Database Systems: A Practical Approach to Design, Implementation, and Management}. Pearson, 6th ed., 2014.

\bibitem{elmasri2016fundamentals}
R. Elmasri e S. B. Navathe, \textit{Fundamentals of Database Systems}. Pearson, 7th ed., 2016.

\bibitem{date2019introduction}
C. J. Date, \textit{An Introduction to Database Systems}. Addison-Wesley, 8th ed., 2019.

\bibitem{loshin2012bigdata}
D. Loshin, \textit{Big Data Analytics: From Strategic Planning to Enterprise Integration}. Morgan Kaufmann, 2012.

\bibitem{liskov2012query}
B. Liskov, "Query Processing and Optimization," in \textit{Principles of Database and Knowledge-Base Systems}, Morgan Kaufmann, 2012.

\bibitem{korth2010database}
H. F. Korth, A. Silberschatz, e S. Sudarshan, \textit{Database System Concepts}. McGraw-Hill, 6th ed., 2010.

\bibitem{anderson2001security}
R. Anderson, \textit{Security Engineering: A Guide to Building Dependable Distributed Systems}. Wiley, 2001.

\bibitem{ferraiolo2003role}
D. Ferraiolo, D. R. Kuhn, e R. Sandhu, \textit{Role-Based Access Control}. Artech House, 2003.

\bibitem{sandhu1996role}
R. Sandhu, E. J. Coyne, H. L. Feinstein, e C. E. Youman, "Role-Based Access Control Models," \textit{IEEE Computer}, vol. 29, no. 2, pp. 38-47, 1996.

\bibitem{menezes1996handbook}
A. J. Menezes, P. C. van Oorschot, e S. A. Vanstone, \textit{Handbook of Applied Cryptography}. CRC Press, 1996.

\bibitem{halfond2006classification}
W. G. J. Halfond, J. Viegas, e A. Orso, "A Classification of SQL Injection Attacks and Countermeasures," in \textit{IEEE International Symposium on Secure Software Engineering}, 2006.

\bibitem{bishop2003computer}
M. Bishop, \textit{Computer Security: Art and Science}. Addison-Wesley, 2003.

\bibitem{Sandhu1996}
R. Sandhu, E. J. Coyne, H. L. Feinstein, e C. E. Youman, "Role-Based Access Control Models," \textit{IEEE Computer}, vol. 29, no. 2, pp. 38-47, 1996.

\bibitem{feldman2014practical}
M. Feldman, \textit{Practical API Design: Confessions of a Java Framework Architect}. Apress, 2014.

\bibitem{kleppmann2017designing}
M. Kleppmann, \textit{Designing Data-Intensive Applications: The Big Ideas Behind Reliable, Scalable, and Maintainable Systems}. O'Reilly Media, 2017.

\bibitem{fielding2000architectural}
R. T. Fielding, "Architectural Styles and the Design of Network-based Software Architectures," Ph.D. dissertation, University of California, Irvine, 2000.

\bibitem{tanenbaum2007distributed}
A. S. Tanenbaum e M. Van Steen, \textit{Distributed Systems: Principles and Paradigms}. Pearson Prentice Hall, 2nd ed., 2007.

\bibitem{resnick2012building}
M. Resnick, \textit{Building Virtual Communities: Learning and Change in Cyberspace}. Cambridge University Press, 2012.

\bibitem{bhattacharya2017web}
S. Bhattacharya, \textit{Web Development with HTML5, CSS, and JavaScript}. Oxford University Press, 2017.

\bibitem{shklar2009enterprise}
L. Shklar e R. Rosen, \textit{Web Application Architecture: Principles, Protocols and Practices}. Wiley, 2nd ed., 2009.

\bibitem{mclean2015interactive}
A. McLean, \textit{Interactive Data Visualization for the Web: An Introduction to Designing with D3}. O'Reilly Media, 2015.

\bibitem{kumar2013computer}
D. Kumar, \textit{Computer Graphics: Principles and Practice}. Pearson, 3rd ed., 2013.

\bibitem{freeman2014head}
E. Freeman e E. Robson, \textit{Head First JavaScript Programming: A Brain-Friendly Guide}. O'Reilly Media, 2014.

\bibitem{mdn_docs}
MDN Web Docs, "MDN Web Docs on HTML, CSS, JavaScript," disponibile su https://developer.mozilla.org, accesso effettuato nel 2023.

\bibitem{flanagan2011javascript}
D. Flanagan, \textit{JavaScript: The Definitive Guide}. O'Reilly Media, 6th ed., 2011.

\bibitem{robbins2012learning}
J. N. Robbins, \textit{Learning Web Design: A Beginner's Guide to HTML, CSS, JavaScript, and Web Graphics}. O'Reilly Media, 4th ed., 2012.

\bibitem{meyer2007css}
E. A. Meyer, \textit{CSS: The Definitive Guide}. O'Reilly Media, 3rd ed., 2007.

\bibitem{marcotte2011responsive}
E. Marcotte, \textit{Responsive Web Design}. New York: A Book Apart, 2011.

\bibitem{marcotte2011responsive}
E. Marcotte, \textit{Responsive Web Design}. New York: A Book Apart, 2011.

\bibitem{flanagan2011javascript}
D. Flanagan, \textit{JavaScript: The Definitive Guide}. O'Reilly Media, 6th ed., 2011.

\bibitem{wroblewski2011web}
L. Wroblewski, \textit{Mobile First}. New York: A Book Apart, 2011.

\bibitem{esposito2020learning}
D. Esposito, \textit{Learning PHP, MySQL \& JavaScript: With jQuery, CSS \& HTML5}. O'Reilly Media, 5th ed., 2020.

\bibitem{mccool2012php}
C. McCool, \textit{PHP Programming with MySQL: The Web Technologies Series}. Cengage Learning, 2nd ed., 2012.

\bibitem{krzywinski2010data}
M. Krzywinski, J. Schein, I. Birol, et al., "Circos: An Information Aesthetic for Comparative Genomics," \textit{Genome Research}, vol. 19, no. 9, pp. 1639-1645, 2010.

\bibitem{silberschatz2020database}
A. Silberschatz, H. F. Korth, e S. Sudarshan, \textit{Database System Concepts}. McGraw-Hill, 7th ed., 2020.

\bibitem{kevin2020flexbox}
K. Powell, \textit{Mastering CSS Flexbox: A Comprehensive Guide to Modern Layouts}. Independently published, 2020.

\bibitem{bonneau2012quest}
J. Bonneau, C. Herley, P. C. van Oorschot, and F. Stajano, \textit{The quest to replace passwords: A framework for comparative evaluation of web authentication schemes}. In \textit{2012 IEEE Symposium on Security and Privacy}, IEEE, 2012, pp. 553–567.

\bibitem{ferraiolo2001role}
D. F. Ferraiolo, D. R. Kuhn, and R. Sandhu, \textit{Role-Based Access Control}. Artech House, 2001.

\bibitem{gentry2009fully}
C. Gentry, \textit{A fully homomorphic encryption scheme}. Stanford University, 2009.

\bibitem{nielsen1994}
J. Nielsen, "Usability Engineering," Morgan Kaufmann, San Francisco, 1994.

\bibitem{cooper2014}
A. Cooper, R. Reimann, D. Cronin, and C. Noessel, "About Face: The Essentials of Interaction Design," John Wiley \& Sons, Indianapolis, 2014.

\bibitem{shneiderman2016designing}
B. Shneiderman, C. Plaisant, M. Cohen, and S. Jacobs, "Designing the User Interface: Strategies for Effective Human-Computer Interaction," 6th ed., Pearson, Boston, 2016.

\bibitem{w3c2018}
W3C, "Web Content Accessibility Guidelines (WCAG) 2.1," World Wide Web Consortium, 2018. [Online].

\bibitem{feldmann2002scalability}
A. Feldmann, "Scalable Internet Services," IEEE Internet Computing, vol. 6, no. 3, pp. 99–100, 2002. doi:10.1109/MIC.2002.1003135.

\bibitem{openssl2020}
OpenSSL Project, "OpenSSL: Cryptography and SSL/TLS Toolkit," 2020. [Online].

\bibitem{monitoring2021}
Z. Wang, Y. Zhu, and K. Yang, "Performance Monitoring in Distributed Systems: Approaches and Frameworks," ACM Computing Surveys, vol. 54, no. 3, 2021, doi:10.1145/3446370.

\bibitem{sandhu1996role}
R. S. Sandhu, E. J. Coyne, H. L. Feinstein, and C. E. Youman, "Role-Based Access Control Models," \textit{IEEE Computer}, vol. 29, no. 2, 1996, doi:10.1109/2.485845.

\bibitem{davidson2016role}
I. Davidson and J. Passmore, "Role-Based Access Control: Improving Security and Compliance in Complex Systems," \textit{International Journal of Security Studies}, vol. 10, no. 4, 2016, doi:10.1007/s10207-016-0043-7.

\bibitem{wright2017rbac}
C. Wright and M. Carlson, "Role-Based Access Control (RBAC) and Error Reduction: Ensuring Security and Reliability in Data-Driven Applications," \textit{Journal of Information Security}, vol. 12, no. 2, 2017, pp. 87-96, doi:10.1016/j.jinfosec.2017.04.004.

\bibitem{gdpr2016regulation}
European Union, "Regulation (EU) 2016/679 of the European Parliament and of the Council of 27 April 2016 on the protection of natural persons with regard to the processing of personal data and on the free movement of such data (General Data Protection Regulation)," \textit{Official Journal of the European Union}, vol. L119, 2016, pp. 1-88.
\end{thebibliography}


	
	\nocite{*}

	%\cleardoublepage
	%\addcontentsline{toc}{chapter}{Bibliografia}


	
	%\listoffigures
	%\listoftables


\end{document}
