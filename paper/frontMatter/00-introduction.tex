%
%%%%%%%%%%%%%%%%%%%%%%%%%%%%%%%%%%%%%%%%
\pagenumbering{roman}                   %serve per mettere i numeri
%%%%%%%%%%%%%%%%%%%%%%%%%%%%%%%%%%%%%%%%%imposta l'intestazione di pagina
\rhead[\fancyplain{}{\bfseries
INTRODUZIONE}]{\fancyplain{}{\bfseries\thepage}}
%%%%%%%%%%%%%%%%%%%%%%%%%%%%%%%%%%%%%%%%%aggiunge la voce Introduzione
                                        %   nell'indice
\chapter*{Introduzione}
\addcontentsline{toc}{chapter}{Introduzione}
La crescente disponibilità di dati pubblici, spesso definiti come \textit{open data}, rappresenta oggi una risorsa fondamentale per promuovere trasparenza, partecipazione civica e innovazione. In ambito urbano, i dati pubblici sono utilizzati per migliorare l’efficienza dei servizi, ottimizzare la mobilità e pianificare politiche ambientali e sociali basate su evidenze concrete. Tuttavia, l’accesso e l’utilizzo efficace di queste informazioni richiedono strumenti tecnologici avanzati che possano rendere i dati fruibili e comprensibili anche per utenti non esperti. In tale contesto si inserisce il progetto di questa tesi, che si propone di creare una piattaforma interattiva per la gestione e visualizzazione dei dati pubblici del Comune di Bologna.

Questa piattaforma è concepita per supportare utenti con diversi livelli di competenza, dai cittadini interessati a comprendere le dinamiche della loro città, ai ricercatori e sviluppatori alla ricerca di informazioni approfondite per analisi o applicazioni specifiche. La piattaforma permette l’accesso a vari dataset tramite una dashboard intuitiva, che presenta funzionalità avanzate di visualizzazione interattiva e di analisi. Gli utenti possono esplorare le informazioni in modo dinamico tramite grafici, mappe e tabelle, rendendo più agevole l'individuazione di pattern e trend.

Il frontend è stato progettato seguendo i principi del design \textit{user-centered}, con particolare attenzione all’accessibilità e alla compatibilità cross-device, così da garantire un’esperienza utente ottimale su qualsiasi dispositivo. Il backend, realizzato con linguaggi e tecnologie consolidati, gestisce la logica applicativa e le operazioni di manipolazione dei dati, garantendo sicurezza e affidabilità anche in caso di richieste simultanee da più utenti. Particolare enfasi è stata posta sui meccanismi di autenticazione e autorizzazione, per garantire un accesso sicuro alle diverse funzionalità e proteggere le informazioni sensibili, in conformità con le normative vigenti, come il GDPR.

Inoltre, la piattaforma offre un sistema di gestione degli accessi basato sui ruoli (RBAC), che permette di differenziare i privilegi degli utenti, definendo in modo preciso chi può visualizzare, modificare o gestire i dati. Questa funzionalità è di fondamentale importanza per garantire la sicurezza delle informazioni e la sostenibilità del sistema in contesti di accesso pubblico.

La tesi è strutturata come segue: il primo capitolo fornisce una panoramica sullo stato dell'arte riguardo le piattaforme per la gestione di open data, con un approfondimento sulle sfide tecniche e di design. Il secondo capitolo esplora le tecnologie e i linguaggi utilizzati per sviluppare la piattaforma, mentre il terzo capitolo è dedicato all'analisi delle scelte progettuali per il frontend e il backend e si conclude con una discussione sulle potenziali evoluzioni della piattaforma.

In sintesi, questo progetto rappresenta un contributo alla gestione e valorizzazione degli open data per il Comune di Bologna, fornendo uno strumento capace di rendere accessibili e utili le informazioni pubbliche per una vasta gamma di utenti. La piattaforma si propone non solo come soluzione tecnica, ma come un passo verso una gestione più consapevole e partecipata delle risorse urbane.
